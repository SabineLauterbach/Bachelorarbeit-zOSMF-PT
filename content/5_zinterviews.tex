\section{Interviews}
In diesem Absatz wird zunächst erläutert, auf welcher Grundlage (Kommentar: was heißt hier Grundlage?) die Interviews geführt wurden.
Anschließend werden die Ergebnisse der einzelnen Interviews nach Gruppen aufgeteilt, ausgewertet und interpretiert.
Schließlich wird daraus ein allgemeines Stimmungsbild abgeleitet.

\subsection{Durchführung}
Interviewt wurden jeweils zwei Mitarbeiter der Gruppen, CICS Administration, Db2 Administration, IBM MQ Administration.
Zusätzlich wurde eine Fachberaterin im Bereich Technologiestrategie und ein Mitarbeiter (Kommentar: wirklich nur einer?) des Entwicklerteams der DATEV Rechnungsschreibung befragt.
Für die beiden letztgenannten Interviews waren nur die Fragen 1., 2. und 6. bis 10. des Fragebogens von Relevanz.
Sowohl der Fragenkatalog, als auch die ausgefüllten und digitalisierten Fragebögen sind im Anhang \ref{app:fragen} zu finden.
Vor Durchführung der Interviews, wurde den Teams in getrennten Terminen die Ergebnisse dieser Arbeit vorgestellt.
Der Schwerpunkt der Vorstellung wurde an das jeweilige Team angepasst.
So wurde bei den Administratorenteams vor allem auf die Erstellung der Templates, was für ihr Arbeitsgebiet relevant ist, eingegangen.
Außerdem wurde neben der im Absatz \ref{sec:akttemp} dargestellten Lösung (Kommentar: welche?), auch die Lösung aus Kapitel \ref{ch:ausblick} (Kommentar: welche?) vorgestellt.
An Hand des durch diese Arbeit bereit gestellten Templates wurde die z/OSMF Oberfläche erläutert.

\subsubsection{CICS Administratoren}
Die Einschätzung des Werts des aktuell möglichen Ablaufs mittels z/OSMF von CICS Administrator 1 ist mittelmäßig.
So bietet es zwar eine flexible Versionierung und Veröffentlichung, jedoch ist es durch die verschiedenen Sprachen und Dokumentarten (Kommentar: siehe ....  einfügen)  mit Startschwierigkeiten versehen. (Kommentar: vlt. eher etwas spezifischer, EInarbeitungsaufwand??)
Im Gegensatz dazu sieht CICS Administrator 2 das momentane Template zumindest für CICS als ablauffähig und mehrfach einsetzbar. (Kommentar: das ist kein Gegensatz zu Admin1, vlt. eher: Dagegen bewertet CICS Admin 2 das Template...... )
Beide Administratoren sehen einen Vorteil in der z/OSPT Lösung, d.h. der Konfigurierbarkeit  außerhalb des Templates 
Als Nachteil bewerten sie jedoch die Notwendigkeit eines sehr dynamischen Templates (KOmmentar: der Nachteil ist nicht die Notwendigkeit, sondern, dass das Template durch die notwendige Komplexität sehr komplex wird.
Die Benutzerfreundlichkeit der Oberfläche wird von CICS Administrator 1 für beide Lösungen (z/OSMF Und z/OSPT) mit sehr gut bewertet.
Auf Grund dessen, dass er noch nicht damit gearbeitet hat, enthält sich CICS Administrator 2 hier der Bewertung.
Die Arbeitsweise bei Änderungen an den Workflow Definitionsdateien und dahinterliegenden Skripten usw., wird als schlecht bis mittelmäßig eingeschätzt.
Hier fehlt beiden Befragten eine geeignete Toolunterstützung, z.B. auch mit Syntaxhighlighting.
Der erste Eindruck wird von einem hohen Ersteinrichtungsaufwand und einer zeitaufwändigen Einarbeitungsphase geprägt.
Zusammen mit den verschiedenen Sprachen hat dies auf CICS Administrator 1 eine eher abschreckende Wirkung.
Dennoch können sich beide Befragten vorstellen, nachdem der Einarbeitungsaufwand erbracht wurde, täglich mit dem Toolkit zu arbeiten, da bei dem aktuell etablierten Prozess ein hoher manueller Aufwand zu erbringen ist und eine Abstimmung zwischen den Administratoren und dem Entwicklerteam notwendig ist.

Zusammenfassend lässt sich sagen, dass die CICS Administratoren eine Chance auf Verbesserung des Prozesses durch das Toolkit sehen.
Allerdings schreckt der hohe Einarbeitungsaufwand und die Mischung aus verschiedenen Sprachen und fehlendem Syntaxhighlighting bei der Templateerstellung ab.

\subsubsection{Db2 Administratoren}
Sowohl Db2 Administrator 1 als auch Db2 Administrator 2 erkennen die Möglichkeit einer Verbesserung durch z/OSMF.
Jedoch sind sie der Meinung, dass noch einiges an Forschung und Weiterentwicklung notwendig ist um es sinnvoll nutzen zu können.
Sie stimmen auch bei ihrer Ansicht bezüglich z/OSPT überein.
So sehen sie den Vorteil des Kommandozeileninterfaces vor allem bei einer Endausbaustufe mit automatisiertem Deployment innerhalb einer CI/DC-Pipline und dem Einsatz von Jenkins basierten Builds und Tests.
Db2 Administrator 2 stört sich an den Begriffen \glqq Container\grqq{} und \glqq Image\grqq, da diese teilweise vertauscht und synonym verwendet werden. (Kommentar: von wem? Von Dir oder von IBM?)
Bezüglich der Benutzerfreundlichkeit der Oberfläche fällt die Bewertung bei beiden schlecht bis mittelmäßig aus.
Db2 Administrator 1 empfindet die gezeigte Arbeitsweise für Änderungen an den Workflow Definitionsdateien als sehr schlecht, da es momentan ohne automatisches Deployment realisiert ist.
Die Bewertung von Db2 Administrator 2 ist mittelmäßig, da eine Entwicklungsumgebung sinnvoll wäre, vor allem in Hinblick auf eine Syntaxprüfung.
Der erste Eindruck des Toolkits ist sehr positiv. (Kommentar: damit ist wieder z/OSPT gemeint, oder?)
Es wird als mächtiges Tool und als Zukunft des modernen Deployments auf dem Mainframe eingeschätzt, jedoch wird es als sehr komplex betrachtet.
Im Vergleich dazu wird der aktuell etablierte Bereitstellungsprozess ebenfalls als komplex beschrieben.
Dieser funktioniere zwar sehr gut, aber es sind viele Abhängigkeiten zu anderen Personen vorhanden, dadurch entstehen Wartezeiten.
Außerdem sei ein sehr umfangreiches Wissen über alle beteiligten Subsysteme notwendig.
Hinzu kommt ein hoher Konfigurationsaufwand und viel Vorarbeit, zum Beispiel Funktionsuser und ein Rechtekonzept.
Beide Db2 Administratoren könnten sich  vorstellen mit dem Toolkit täglich zu arbeiten, um diese Probleme anzugehen.
Eine INtegration mit einer Jenkins-basierten Build-Pipeline wird hierfür von Db2 Administrator 1 vorausgesetzt.

Die Interviews mit den Db2 Administratoren ergaben folgendes Bild.
Sie sehen in dem Toolkit eine gute Möglichkeit, um den Bereitstellungsprozess zu vereinfachen und weniger zeitaufwändig zu gestalten.
Allerdings ist noch viel Forschungsarbeit in dieses Thema zu investieren.
Als Hauptpunkt ist die Nutzung von Jenkins und damit die Einbindung und Etablierung einer automatisierten Build-Pipeline zu nennen.

\subsubsection{IBM MQ Administratoren}
IBM MQ Administrator 1 sieht bereits im aktuell funktionsfähigen Template einen Mehrwert.
Zum einen, weil mehr Verantwortung im Entwicklerteam liegt und zum anderen sind weniger manuelle Eingriffe bei der Bereitstellung durch die MQ-Administrtion notwendig.
Jedoch ist die Lösung, die im Ausblick gezeigt wurde (Kommentar welche? ...) , flexibler und damit etwas besser geeignet.
Zudem seien die momentan bereits vorhandenen Features (Kommentar: welche? Meinst du den momentan vorhandenen Prozess???) durchaus gut, jedoch kam die Frage auf, ob die IBM das `IBM Cloud Provisioning and Management for z/OS`-Toolkit noch weiterentwickeln wird.
Die Benutzerfreundlichkeit der z/OSMF Oberfläche wurde als mittelmäßig bis gut eingestuft.
Bezüglich des Arbeitens, Verwaltens und Änderns von Workflow Definitionsdateien und der dazugehörigen Skripte konnte keine Bewertung abgegeben werden, da noch nicht selbst damit gearbeitet wurde (Kommentar: DU hast es ihnen ja gezeigt, d.h. wenn dann müsstest du schreiben, wollten die Kollegen nur aufgrund der Demo und ohne selbst damit gearbeitet zu haben, keine Bewertung abgeben)
Dies hat auch Einfluss auf den ersten Eindruck. 
So wird zu bedenken gegeben, dass der Zeitaufwand und die zu leistenden Vorarbeiten mit einzubeziehen (Kommentar: in was? in d ie Bewertung? sind.
Vor allem, wenn die Provisionierung von einem IBM MQ Queue Manager hinzu kommt (Kommentar: wo hinzu, die aktuell noch Manuelle Bereitstellung oder die, die noch im Toolkit dazu kommt? Missverständlich).
Jedoch kann sich IBM MQ Administrator 1 vorstellen mit dem Toolkit täglich zu arbeiten, da letztendlich die Werkzeugwahl keine Rolle spielt. (Kommentar: welche Werkzeugauswahl? Bei was spielt sie keine Rolle?)
Diese Einschätzung  wird dadurch begünstigt, dass der aktuell etablierte Prozess schlecht beurteilt wird, aufgrund des hohen manuellen Aufwands und der dadurch erzeugten Rückfragen mit den Beteiligten.
Zuletzt wird noch darauf hingewiesen, dass das Toolkit generell noch Neuland sei.
So müssten erst die Grundlagen gelernt und damit Erfahrungen gesammelt werden bevor eine qualifizierte Bewertung möglich sei. (Kommentar: das kommt nicht so gut rüber, so als ob du es ihnen noch nicht gut genug gezeigt hättest. Vlt. eher die  "endgültige qualifizierte Bewertung)

Im Vergleich zu IBM MQ Administrator 1 fehlen IBM MQ Administrator 2 noch weitere Automatismen.
So sind trotz des Einsatzes von z/OSPT noch Absprachen mit Dritten, wie dem RACF-Team und dem Speicher-Team, notwendig.
z/OSPT sei zudem zwar "Docker ähnlich", ist jedoch keine vollumfängliche Containerlösung.
So könnte sich IBM MQ Administrator 2 zwar vorstellen, dass Entwickler dem Toolkit täglich  arbeiten, aber es müsste ohne manuelle Eingriffe funktionieren. (Kommentar: was heißt hier täglich arbeiten... das erstellen eines Toolkit-Templates kann nicht ohne manuelle Eingriffe funktionieren... Das ist ja der Job der Admins. Die manuellen Eingriffe werden für den Entwickler wegfallen, nicht für den Admin)
Die Erstellung der Skripte muss mit einem einmaligen Aufwand verbunden sein, so dass sie keine ständigen Anpassungen benötigen.
Davon wird auch der erste Eindruck beeinflusst.
Insgesamt sind seiner Meinung nach zwar viele gute Ansätze vorhanden, aber es fehlen Analogien und eine Ähnlichkeit zu Jenkins-Skripten, die mit groovy, yaml oder ansible playbooks arbeiten, XML "sei nicht mehr zeitgemäß". 

Zusammenfassend lässt sich sagen, dass sich beide IBM MQ Administratoren einig sind, dass der momentan etablierte Bereitstellungsprozess nicht mehr gut genug (Kommentar: passt das?) ist und ein neuer Prozess durchaus notwendig wäre.
Der durch diese Arbeit gezeigte Prozess als Ablöse wird als prinzipiell möglich erachtet, jedoch nur der Einsatz mit z/OSPT.
Außerdem wird vor einer hohen Lernkurve und der noch fehlenden Automation, sowie der fehlenden Ähnlichkeit zu Jenkins oder anderen PaaS Lösungen (Kommentar: nochmal: welche?) gewarnt.

\subsubsection{Entwicklerteam der DATEV Rechnungsschreibung}
Aus Sicht des Entwicklers wird für den gezeigten Bereitstellungsprozess viel Wissen über die z/OSMF Oberfläche und das Template selbst benötigt. (Kommentar: eigentlich muss er das ja nicht wissen. d.h. wenn er den Eindruck hat, dann wäre die Demo nicht gut gewesen...)
Dieses Wissen müsse auch bei nicht häufiger Nutzung über einen längeren Zeitraum erhalten werden.
Der Prozess sei zwar schon "ganz gut", jedoch sei weiterhin viel Absprache mit den Administratoren notwendig. (Kommentar: ist das so?)
Hier wird auch der Nachteil des momentan etablierten Bereitstellungsprozesses gesehen.
Jedoch sobald der Erstaufwand für die Einarbeitung in z/OSMF geleistet wurde, muss sich nur im Ausnahmefall noch darum gekümmert werden.
Diese Verantwortung würde im Fall der Umsetzung mit z/OSPT bei dem Entwicklerteam selbst liegen. (Kommentar: das ist echt schwierig zu verstehen, lass uns das morgen mal umformulieren auf Basis der Interview-Unterlagen).
Durch die eingesparten Absprachen erhofft man  sich eine höhere Effizienz im  (KOmmentar: in was? tägliche Arbeit? WEnn das so wäre dann widerspricht das der "nicht häufigen Nutzung oben).
Es wurde noch die Nutzung in der Qualitätssicherungs- und Produktionsstage genannt.
Hier werden Vorteile einer einfache Skalierbarkeit gesehen. (Kommentar: zu oberflächlich: versteht man nicht. für die Zukunft könnte man sich die Nutzung auch für die QS_ und Prod-Stage vorstellen, und dort z.B. auch CICS-Umgebungen horizontal skalieren. Dies ist jedoch nicht Teil der vorliegenden Arbeit).
Vor allem auf Grund der Eigenverantwortung über die Subsysteme (Kommentar: über das Management der eigenen Testssysteme mit den Subsystemen) könnte man sich die tägliche Arbeit mit dem Toolkit vorstellen.
Allerdings nur in Hinblick auf eine Integration in eine Jenkins Build Pipeline, für die erst einmal Sammeln von Erfahrungen bezüglich des Prozesses und des Toolings notwendig wären.

Das Hauptaugenmerk des Entwicklers liegt bei der höheren Eigenverantwortung beziehungsweise der Eigenverwaltung von benötigten Subsystemen.
Eine einfache und intuitive Bedienung des Toolkits ist außerdem wichtig.

\subsubsection{Fachberaterin im Bereich Technologiestrategie}
Laut der Fachberaterin im Bereich Technologiestrategie ist der gezeigte Ablauf beziehungsweise die z/OSMF Oberfläche für eine solche Aufgabe (Kommentar: für die Aufgabe des Provisionierens von z/OS Middleware) geeignet.
Jedoch sei es besser wenn z/OSMF in den bereits existierenden \glqq Marktplatz\grqq{} für DATEV Cloud Lösungen integriert wäre.
Der Prozess, der mit Hilfe von z/OSPT ermöglicht wird, wird als gut angesehen, da durch ihn die Entwicklung von z/OS Anwendungen an die Vorgehensweise der Cloud Native Entwicklung angenähert wird.
Hier kommt die Rolle des Build Engineers auch für solche Anwendungen ins Spiel.
Dieser kümmert sich um die Erstellung und Pflege der Build-Pipeline.
Große Nachteil im momentan etablierten Bereitstellungsprozess sei vor allem,  dass eine Anzahl vpn Entwicklern, die parallel an einem Produkt arbeiten, sich die gleiche Entwicklungssystemumgebung teilen.
So arbeiten alle mit der gleichen CICS-Instanz, der gleichen Test-Datenbank und mit den gleichen IBM MQ Queues.
Dadurch beeinflussen Änderungen des einen Entwicklers die Tests der anderen Kollegen, es entsteht Koordinationsaufwand.
Falls Änderungen an der Umgebung notwendig sind, kann während dieser Zeit kein Entwickler weiterarbeiten.
Hier liege der Vorteil des Toolkits.
Es ermöglicht aus Entwicklersicht eine sehr einfache, schnelle Möglichkeit eine isolierte Umgebung bereitzustellen, unabhängig von den Administratorenteams.
Zusätzlich dienen die Konfigurationsdateien auch als Dokumentation, welche Ressourcen für ein erneutes Erstellen der Umgebung notwendig sind.

Abschließend lässt sich sagen, dass aus Sicht einer Fachberaterin im Bereich Technologiestrategie dieses Toolkit die Entwicklung beziehungsweise den Bereitstellungsprozess deutlich verbessern kann.
So ist für den Entwickler ein an die Cloud Native Welt angenäherter Entwicklungsprozess möglich.
Dadurch wird der Wechsel zwischen beiden Umgebungen immer fließender.

\subsection{Meinungsbild}
Über alle Gruppen hinweg lassen sich folgende Punkte zusammenfassen:

\begin{samepage}
\begin{itemize}
\item neuer Prozess notwendig
\item z/OSPT Lösung bevorzugt
\item erste Erfahrungen sammlen
\end{itemize}
\end{samepage}

Es stimmen alle Gruppen überein, dass der momentan etablierte Bereitstellungsprozess für Mainframesubsysteme durch viele Absprachen und Abstimmungs-Aufwand zeitaufwändig ist.
Sie würden einen neuen, schnelleren Prozess begrüßen. (Kommentar: automatisiert!!)

Jedoch muss dieser Prozess aus Entwicklersicht mit minimalem Konfigurationsaufwand verbunden sein.
Dies könnte durch eine Provisionierung mittels z/OSPT und einer  Integration in eine Jenkins Build Pipeline oder durch die Einbindung in den \glqq DATEV Marktplatz\grqq{} mittels eines entwickelten \glqq Service Brokers\grqq. gewährleistet werden.
Aus Sicht der Administratoren sind mit dieser Umsetzung nur wenige allgemeine Templates zu verwalten, da die Entwickler mit z/OSPT Images und keine weiteren Templates erzeugen.
Um diese Punkte zu ermöglichen, muss das Template umgestaltet werden.
Der dadurch in den Administratorenteams entstehende Aufwand und die damit verbundene steile Lernkurve hat eine abschreckende Wirkung.

Trotz dieser abschreckenden Wirkung sind auch die Administratorenteams bereit, falls die Kapazitäten vorhanden sind, den Bereitstellungsprozess mit Hilfe des \glqq IBM Cloud Provisioning and Management for z/OS\grqq{} zu verbessern.
Aus Sicht der Technologiestrategie ist dies ein wichtiger und notwendiger Schritt hin zu einem Cloud Native ähnlichen Prozess.
