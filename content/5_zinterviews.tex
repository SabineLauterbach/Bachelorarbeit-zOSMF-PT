\section{Interviews}
In diesem Absatz wird zunächst erläutert, auf welcher Grundlage die Interviews geführt worden sind.
Anschließend werden die Ergebnisse der einzelnen Interviews nach Gruppen aufgeteilt, ausgewertet und interpretiert.
Schließlich wird daraus ein allgemeines Stimmungsbild abgeleitet.

\subsection{Durchführung}
Interviewt wurden jeweils zwei Mitarbeiter der Gruppen, CICS Administration, Db2 Administration, MQ Administration und Entwicklerteam der Rechnungsschreibung.
Zusätzlich wurde ein Mitarbeiter aus dem Architekturstrategieteam befragt.
Sowohl der Fragenkatalog als auch die ausgefüllten und digitalisierten Fragebögen sind im Anhang \ref{app:fragen} zu finden.
Bevor die Interviews durchgeführt worden sind, wurde den Teams in getrennten Terminen die Ergebnisse dieser Arbeit vorgestellt.
Der Schwerpunkt wurde an das jeweilige Team angepasst.
So wurde bei den Administratorenteams vor allem auf den Teil des Templates, der für ihr Arbeitsgebiet zuständig ist, eingegangen.
Außerdem wurden neben den im Absatz \ref{sec:akttemp} dargestellten Lösung, auch die Lösung aus Absatz \ref{sec:techaus} vorgestellt.
An Hand des durch diese Arbeit bereit gestellte Template wurde die zOSMF Oberfläche erläutert.

\subsubsection{CICS Administratoren}
\subsubsection{Db2 Administratoren}
\subsubsection{MQ Administratoren}
MQ Administrator 1 sieht bereits im aktuell funktionsfähigen Template einen Mehrwert.
Zum einen weil mehr Verantwortung im Entwicklerteam liegt und zum anderen sin weniger händische Eingriffe notwendig.
Jedoch ist die Lösung, die im Ausblick gezeigt wurde, flexiber und damit etwas besser geeignet.
Zudem seien die momentan bereits vorhanden Features durchaus gut, jedoch kam die Frage auf, ob die IBM das `IBM Cloud Provisioning and Management for z/OS`-Toolkit noch weiterentwickelt.
Die Benutzerfreundlichkeit der zOSMF Oberfläche wurde als mittelmäßig bis gut eingestuft.
Bezüglich des Arbeiten, Verwalten und Ändern von Workflow Definitionsdateien und den dazugehörigen Skripten konnte keine Bewertung abgegeben werden, da noch nicht selbst damit gearbeitet wurde.
Dies hat auch Einfluss auf den ersten Eindruck.
So wird zu bedenken gegeben, dass der Zeitaufwand und die zu leistenden Vorarbeiten mit einzubeziehen sind.
Vor allem, wenn die Provisionierung von einem MQ Queue Manager hinzu kommt.
Jedoch kann sich MQ Administrator 1 vorstellen mit dem Toolkit täglich zu arbeiten, da letztenendlich die Werkzeugwahl keine Rolle spielt.
Diese Entscheidung wird dadurch begünstigt, dass der aktuell etablierte Prozess schlecht beurteilt wird.
Aufgrund des hohen manuellen Aufwands und der dadurch erzeugten Rückfragen.
Zuletzt wird noch darauf hingewiesen, dass das Toolkit generell noch Neuland sei.
So müssten erst die Grundlagen gelernt und damit Erfahrung gesammelt werden bevor eine qualitativere Bewertung möglich sei.
Dies beinhaltet wahrscheinlich eine starke Lernkurve.

Im Vergleich zu MQ Administrator 1 fehlt MQ Administrator 2 noch weitere Automatismen.
So sind trotz des Einsatzen von zOSPT noch Absprachen mit Dritten, wie dem RACF-Team und dem Speicher-Team, notwendig.
zOSPT sei zudem nur Docker ähnlich, ist jedoch keine vollumfängliche Containerlösung.
So könnte sich MQ Administrator 2 zwar vorstellen mit dem Toolkit täglich zu arbeiten, aber es müsste ohne manuelle Eingriffe funktionieren.
Die Erstellung der Skripte muss mit einem einmaligen Aufwand verbunden sein, so dass sie keine ständigen Anpassungen benötigen.
Davon wird auch der erste Eindruck beeinflusst.
So sind zwar viele gute Ansätze vorhanden, aber es fehlen Analogien und eine Ähnlichkeit zu Jenkins und anderen PaaS Lösungen.
Dies geht soweit, dass XML nicht mehr als zeitgemäß betrachtet wird, sondern auf Umsetzungen in groovy, yaml oder mit ansible playbooks zu setzen sei.

Zusammenfassend lässt sich sagen, dass sich beide MQ Administratoren einig sind, dass der momentan etablierte Bereitstellungsprozess schlecht ist und ein neuer Prozess durchaus notwendig wäre.
Der durch diese Arbeit gezeigte Prozess als Ablöse wird prinzipiell als möglich erachtet.
Jedoch nur der Einsatz mit zOSPT.
Außerdem wird vor einer starken Lernkurve und noch fehlender Automation und der damit einhergehenden Ähnlichkeit zu Jenkins oder anderen PaaS Lösungen gewarnt.

\subsubsection{Entwicklerteam der Rechnungsschreibung}

\subsubsection{Architekturstrategieteam}
