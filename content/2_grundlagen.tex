\chapter{Grundlagen}\label{ch:grundlagen}
In diesem Kapitel werden für diese Arbeit wichtige Begriffe und Systeme erläutert.

\section{Customer Information Control System}\label{cics}
Das Customer Information Control System, kurz CICS, ist ein Applikationsserver für einen IBM-Großrechner.
Ein Applikationsserver stellt eine Umgebung zur Verfügung, in der Anwendungen gehostet werden können.
Dabei kümmert sich dieser unter anderem um Transaktionalität, Webkommunikation und Sicherheit.
Hierfür stellen Applikationsserver eine API zur Verfügung.
CICS hat einen weiteren Vorteil, es unterstützt verschiedene Programmiersprachen.
So können Programme innerhalb einer Anwendung in der für ihren Use-Case am besten geeigneten Sprache implementiert werden.
Zu den unterstützten Sprachen zählen neben COBOL und IBM Assembler auch Java und Java EE.
\cite{Rayns.2011}

\subsection{CICS Transaktion}\label{subsec:trans}
Ein Businessablauf wird im CICS in einer Transaktion gekapselt.
So kann eine Transaktion mehrere Programme unterschiedlicher Programmiersprachen umfassen.
Eine Transaktion besitzt ein eindeutiges Kürzel, die TransaktionsID.
Über die TransaktionsID kann der Ablauf gestartet werden.
Dies kann sowohl per Webanfrage oder per Messaging Queue als auch aus einem anderem Programm heraus oder per Hand geschehen.
In der Transaktion werden alle Änderungen die Programme an Resourcen, wie zum Beispiel einer Datenbank oder Dateien, tätigen protokolliert.
So wird im Fehlerfall sichergestellt, dass diese rückgängig gemacht werden können.
 \cite{Rayns.2011}

\subsection{Voraussetzungen}\label{subsec:voraus}
Im Umfeld der DATEV eG sind die Hard- und Softwarevorausetzungen um ein CICS beziehungsweise eine CICS Instanz zu erstellen und zu starten vorhanden.
Der Fokus dieser Arbeit liegt auf letzterem somit werden nur die dafür notwendigen Voraussetzungen dargelegt.
Außerdem liegt der Fokus nur auf Systemen, die vorerst nicht für die produktiven Systeme der DATEV eG vorgesehen sind.
Aus diesem Grund werden nur Schritte, die für ein solches Testsystem benötigt werden, dargestellt.
Eine weitere Eingrenzung besteht darin, dass nur die Arbeitsschritte, die mit z/OSMF\footnote{Beschreibung in Absatz \ref{sec:zosmf}} automatisiert werden, erläutert werden.

\subsection{Einrichtung CICS Instanz}\label{subsec:createCICS}
Die in diesem Absatz benötigten Informationen stammen aus Gesprächen mit Mitarbeiter 2 aus der Abteilung, die für die CICS Administration zuständig ist.
Um eine lauffähige CICS Instanz den Vorausetzungen aus dem Absatz \ref{subsec:voraus} entsprechend einzurichten, sind mehrere Schritte notwendig.
Diese werden im Folgendem beschrieben.

\subsubsection{CICS spezifische Dateien}\label{sssec:speziDat}
Zunächst müssen CICS spezifische Dateien im z/OS angelegt werden.
Im Falle dieser Arbeit handelt es sich um 17 verschiedene.
Diese Dateien benötigt die CICS Instanz um zum Beispiel Systemfehler zu protokollieren.
Eine weitere Datei ist dafür zuständig, dass ein Debugger innerhalb der Instanz verwendet werden kann.

\subsubsection{CSD}
In der CICS system defintion, kurz CSD, Datei muss jede Ressource, die dem System zur Verfügung stehen soll, definiert werden.
Eine CSD Datei kann für mehrere CICS Instanzen verwendet werden.
Eine solche allgemeine CSD Datei hat ca. 22.600 Einträge.
Dort ist unter anderem für jede CICS Instanz hinterlegt, zu welchem Db2 Datenbanksystem und welchem MQ Messagingsystem sich diese Instanz verbinden soll.

\subsubsection{STC Job}
Bei einem Started Task Controll-Job, kurz STC Job, handelt es sich um einen Batch Job, der mit Hilfe des `START`-Konsolenkommandos innerhalb von z/OS gestartet werden kann.
Dieser Batch Job wird deshalb auch als Started Task bezeichnet.\cite{Cassier.2007}
Bei der DATEV eG existiert für jedes CICS ein solcher Job.
In diesem werden zunächst einige zur Laufzeit benötigten Bibliotheken und Dateien eingebunden, unter anderem die CICS spezifischen Dateien\footnote{Beschreibung in Absatz \ref{sssec:speziDat}}.
Außerdem werden hier die SIT \footnote{CICS system initialization table} Parameter definiert.
Zunächst wird festgelegt welche Standard SIT verwendet werden soll.
Anschließend können diese Standardwerte überschrieben werden.
Zu diesen Parameter zählen unter anderem, der eindeutige Name der CICS Instanz, der Speicherort der dazugehören CSD und ob eine Verbindung zu einem Db2 Datenbanksystem hergestellt werden soll.

\subsection{Entfernung CICS Instanz}
Um eine CICS Instanz zu entfernen muss diese zunächst gestoppt werden.
Dies ist über das `STOP`-Konsolenkommando von z/OS möglich.
Anschließend müssen alle im Absatz \ref{subsec:createCICS} beschriebene Schritte rückgängig gemacht werden.
Also müssen die für diese Instanz spezifischen Dateien, die Einträge für die CICS Instanz aus der CSD Datei und schließlich auch der STC Job gelöscht werden.

\section{IBM Cloud Provisioning and Management for z/OS}\label{sec:zosmf}
In diesem Absatz wird zunächst auf die für dieses Kapitel grundlegenden Begriffe eingegangen.
Anschließend wird IBM Cloud Provisioning and Management for z/OS, kurz z/OSMF, erläutert.
Im Anschluss darauf wird das auf Kommandozeilenbefehle basierende z/OS Provisioning Toolkit, kurz z/OS PT, und dessen Möglichkeiten dargestellt.

\subsection{Begrifferklärung}
Im folgenden werden einige allgemeine Begriffe, die im Umfeld von IBM Cloud Provisioning and Management for z/OS vorkommen, erläutert.

\subsubsection{Workflow}
Ein Workflow ist eine beliebig komplexe eindeutige Aneinanderreihung von sogenannten Steps.
Nach der Ausführung dieser wird ein bestimmtes Ziel erreicht, zum Beispiel die erfolgreiche Bereitstellung eines CICS Systems.
Die Definition eines Workflows und den dazugehörigen Steps wird in XML umgesetzt.
Ein Step beschreibt einen Teilablauf eines Workflows.
Innerhalb eines Steps können sowohl interne und externe Scripte als auch JCLs und somit Programme ausgeführt werden.
Des weiteren besteht die Möglichkeit REST-Calls zu tätigen.
Außerdem können Bedingungen für die Durchführung eines Steps definiert werden.
So ist es zum Beispiel möglich einen Step nur durchzuführen, wenn eine bestimmte Variable einen bestimmten Wert besitzt.
Ein weiteres Beispiel ist, es können erforderliche Steps definiert werden, so dass bevor ein Step auf eine Datei zugreift, mittels eines vorherigen Steps geprüft wird ob diese vorhanden ist und wenn nicht diese erzeugt.
Es wird ein XML Schema verwendet um sicherzustellen, dass zur Laufzeit keine syntaktischen Fehler vorhanden sind.
\cite{Rotthove.2018}

Ein Nachteil von Workflows ist, dass diese statisch sind, das heißt, dass die Variablenzuweißungen immer zum Zeitpunkt der Erstellung stattfindet.
Dadurch ergibt sich, dass für jede kleine Änderung ein eigener Workflow erzeugt werden muss.
Somit ist ein Workflow eher ein Einmal- bzw. Wegwerfprodukt.



\subsection{z/OS Provisioning Toolkit}
z/OS PT bietet ein Kommandozeileninterface zum bereitstellen und verwalten von Laufzeitumgebungen zur Verfügung
\cite{IBM.2019}

Technische Begriffe erklären / Mainframe Begriffe erklären

