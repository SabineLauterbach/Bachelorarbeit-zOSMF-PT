\chapter{Grundlagen}\label{ch:grundlagen}
In diesem Kapitel werden für diese Arbeit wichtige Begriffe erläutert.

\section{Mainframe / Großrechner}\label{sec:mainframe}
Im modernen Sprachgebrauch kann ein Großrechner oder auch Mainframe als größte zur Verfügung stehende Serverart betrachtet werden.
Er wird von Unternehmen verwendet, um  kommerzielle Datenbanken, Transaktionsserver und Anwendungen, die einen hohen Grad an Sicherheit und Verfügbarkeit benötigen, zu hosten.
Im Gegensatz zu verteilten Serversystemen, bei denen die Funktionalitäten auf einzelne Server, wie zum Beispiel einen E-Mail-Server, einen Datenbank-Server, einen Web-Server usw. aufgeteilt sind, handelt es sich bei einem Mainframe um ein zentralisiertes System.
Die einzelnen Funktionalitäten werden von sogenannte \glqq Subsysteme\grqq, auch \glqq Middleware\grqq{} genannt, zur Verfügung gestellt.
Darunter zählen unter anderem Datenbanksysteme und Anwendungsserver.
\cite{Ebbers.2011}

\section{Mainframe Anwendungen bei DATEV e.G.}
Das Betriebssystem des IBM Mainframes ist z/OS.
Darauf aufbauend benötigen klassische z/OS Anwendungen bestimmte Middleware.
Bei der DATEV e.G. handelt es sich unter anderem um folgende Middlewarekomponenten:

\begin{itemize}
\item Laufzeitumgebung: CICS oder Batch
\item Datenhaltung: VSAM oder Db2
\item Message Queuing: IBM MQ
\end{itemize}

Diese Subsysteme stehen in jeder Stage zur Verfügung.
Eine Stage beschreibt eine isolierte Systemumgebung mit eigenen Subsystemen und Ressourcenverwaltung.
Die DATEV e.G. unterscheidet vier Stages:
\begin{samepage}
\begin{itemize}
\item Testplex:\\
Labor für Änderungen am System, beispielsweise einer neuen Betriebssystemsversion
\item Entwicklung:\\
Implementierung neuer Features und Durchführung kleiner Tests
\item Qualitätssicherung:\\
Durchführung von Integrationstests
\item Produktion:\\
Software, die für den Kunden bereitsteht
\end{itemize}
\end{samepage}

Betreut werden die einzelnen Subsysteme über alle Stages hinweg von eigens dafür entstandenen Administratorenteams.
Die \glqq CICS Administration\grqq{} kümmert sich um alles rund das CICS Subsystem.
Die \glqq Db2 Administration\grqq{} stellt Datenbanken und Tabellen auf Anfrage der Entwickler bereit.
Die \glqq IBM MQ Administration\grqq{} verwaltet die IBM MQ Ressourcen.

Für die Beantwortung der Forschungsfragen liegt der Fokus auf dem Erstellen (\glqq Provisionieren \grqq) einer anwendungsspezifischen Laufzeitumgebung mit einer Datenhaltung und Message Queuing innerhalb des Testplexes und der Entwicklung.
Als Laufzeitumgebung wird \glqq CICS\grqq, als Datenhaltung \glqq Db2\grqq{} und für das Message Queuing \glqq IBM MQ\grqq{} verwendet.
Diese werden im Folgenden erläutert, hierzu dient Abbildung \ref{fig:archüber} als Überblick.

\begin{figure}[h]
\centering
\includegraphics[width=\textwidth]{figures/architektur.pdf}
\caption{Architekturübersicht über die Subsysteme bei DATEV eG}
\label{fig:archüber}
\end{figure}

\subsection{Customer Information Control System}\label{cics}
Das Customer Information Control System, kurz CICS, ist ein Applikationsserver für einen IBM-Großrechner mit Betriebssystem z/OS und damit eine IBM Middleware.
Ein Applikationsserver stellt eine Umgebung zur Verfügung, in der Anwendungen gehostet werden können.
Dabei kümmert sich dieser unter anderem um Transaktionalität, Webkommunikation und Sicherheit.
Hierfür stellen Applikationsserver eine API zur Verfügung.
CICS hat einen Vorteil gegenüber anderen Anwendungsservern, es unterstützt verschiedene Programmiersprachen.
CICS ist ein Multi-Language Application Server und unterstützt z.B. COBOL, Assembler, Java und PLI.
So können Programme innerhalb einer Anwendung in der für ihren Use-Case am besten geeigneten Sprache implementiert werden.
\cite{Rayns.2011}

Das CICS Subsystem einer Stage umfasst mehrere CICS Instanzen.

\subsubsection{CICS Instanz} 
Unter einer CICS Instanz ist ein einzelner Bereich, der auf dem z/OS Kernel aufsetzt, zu verstehen.
Dieser Bereich ist mittels einer eindeutigen CICS ApplicationID gekennzeichnet und kann darüber explizit angesprochen werden.
Eine CICS Instanz verwaltet mehrere CICS Transaktionen.

Wenn in dieser Arbeit von dem CICS gesprochen wird, ist die CICS-Instanz damit gemeint.

\subsubsection{CICS Transaktion}\label{subsec:trans}
Ein Businessablauf wird im CICS in einer Transaktion gekapselt.
Eine Transaktion kann mehrere Programme unterschiedlicher Programmiersprachen umfassen und wird über eine eindeutige \glqq TransaktionsID\grqq{} identifiziert..

Über die TransaktionsID wird der Ablauf gestartet.
Dies kann sowohl per Webanfrage oder per Messaging Queue als auch aus einem anderen Programm heraus oder manuell geschehen.
In der Transaktion werden alle Änderungen, die Programme an Ressourcen, wie zum Beispiel einer Datenbank oder Dateien tätigen, protokolliert.
So wird im Falle eines Fehlers die Möglichkeit eines Rollbacks sichergestellt.
 \cite{Rayns.2011}

\subsection{Db2}\label{sssec:db2}
Db2 ist ein relationales Datenbanksystem, welches unter anderem als Subsystem eines z/OS Betriebssystems läuft.
Einer Stage können mehrere Datenbanksysteme, auch Instanzen genannt, zugeordnet werden.
In einer Instanz befinden sich die Datenbanken und Tabellen.

\subsection{IBM MQ}\label{sec:mq}
IBM MQ ist eine Messaging-Lösung der IBM.
Diese ermöglicht den asynchronen Datenaustausch zwischen Anwendungen mittels sogenannter Queues.
Alle IBM MQ Begrifflichkeiten, die in dieser Arbeit verwendet werden, werden im Folgenden erläutert.
\cite{Aranha.2013}

Das IBM MQ Subsystem einer Stage setzt sich aus einem oder mehreren Queue Managern zusammen.
Ein Queue Manager kann daher als IBM MQ Instanz gesehen werden.

\subsubsection{Queue Manager}
Bei einem Queue Manager handelt es sich um die zentrale Ressource eines IBM MQ Systems.
Er verwaltet  alle anderen IBM MQ Ressourcen.
Dazu gehören unter anderem die Speichersteuerung der Daten und die Wiederherstellung dieser im Falle eines Fehlers.
Desweiteren koordiniert er den Zugriff aller Anwendungen auf die Nachrichten in den von ihm verwalteten Queues.
Um hierbei die Konsistenz sicherzustellen, sorgt er für Locking und die notwendige Isolation der Queues.
\cite{Aranha.2013}

\subsubsection{Queues}
In Queues werden die Nachrichten, die von Programmen gesendet und gelesen werden gespeichert.
Es gibt verschiedene Arten von Queues, die im Kontext dieser Arbeit relevanten Queues sind folgende:

\paragraph{Die Local Queue.}~\\
Dabei handelt es sich um die einzige Queue Art, bei der die Nachrichten physikalisch gespeichert werden.
Die anderen Queue Arten nutzen als Basis immer eine Local Queue.

\paragraph{Initiation Queue}~\\
Die sogenannte \glqq Initiation Queue\grqq{} ist eine spezielle Art der Local Queue.
Diese dient dem Queue Manager dazu, unter bestimmten Bedingungen eine Trigger-Nachricht darauf zu schreiben.
Daher kann eine andere Local Queue so definiert sein, dass sobald eine Nachricht auf sie geschrieben wird eine solche Trigger-Nachricht erzeugt wird.
Dies ermöglicht, dass Anwendungen nur starten, wenn wirklich Daten zum Verarbeiten vorhanden sind.
\cite{Aranha.2013}

\subsubsection{Process}
Für das Auslösen von Anwendungen wird nicht nur die Initiation Queue benötigt, sondern auch sogenannte \glqq Processes\grqq.
So muss der Local Queue, die den Start einer Anwendung auslösen soll, bei der Definition nicht nur die Initiation Queue bekannt gemacht werden, sondern auch ein Process.
Ein Process legt den \glqq Type\grqq{} und den Namen der zu startenden Anwendung fest.
Als \glqq Type\grqq{} können beispielhaft CICS oder auch WINDOWSNT für Windows unterstütze Plattformen genannt werden.
Ist der \glqq Type\grqq{} CICS,  muss der Name der Transaktion angegeben werden, für Windows Plattformen der Dateipfad der auszuführenden exe.
\cite{Aranha.2013}

\section{IBM Cloud Provisioning and Management for z/OS}\label{sec:zosmf}
In diesem Absatz wird zunächst auf die für dieses Kapitel grundlegenden Begriffe eingegangen.
Das IBM Cloud Provisioning and Management for z/OS bietet die Möglichkeit, mehrere Systeme (sog. Middleware) innerhalb eines z/OS Betriebssystems zu provisionieren, unter anderem Laufzeitumgebungen wie CICS. 
Für diese Aufgaben stehen zwei Schnittstellen zur Verfügung.
Zum einen das z/OS Provisioning Toolkit, (z/OSPT) und zum anderen z/OS Management Facility (z/OSMF).
\cite{KeithWinnardGaryPuchkoffHirenShah.2016}

\subsection{Begrifferklärung}
Im Folgenden werden einige allgemeine Begriffe, die im Umfeld des Tools \glqq IBM Cloud Provisioning and Management for z/OS\grqq{} vorkommen, erläutert.

\subsubsection{Provisioning}
Ins Deutsche übersetzt bedeutet es Bereitstellung,vorliegend wird auch Provisionierung verwendet.
In dieser Arbeit umfasst dieser Begriff die Bereitstellung einer Laufzeitumgebung, beziehungsweise den Prozess, der hierfür benötigt wird.

\subsubsection{Workflow}\label{sssec:workflow}
Ein Workflow ist eine beliebig komplexe, eindeutige Aneinanderreihung von sogenannten Steps.
Nach der Ausführung dieser Steps wird ein bestimmtes Ziel erreicht, zum Beispiel die erfolgreiche Bereitstellung eines CICS Systems.
Für die Definition eines Workflows, der dazugehörigen Steps und ihrer Variablen wird XML genutzt.
Ein Step beschreibt einen Teilablauf eines Workflows.
Innerhalb eines Steps können sowohl interne und externe Scripte als auch JCLs und somit Programme ausgeführt werden.
Darüber hinaus besteht die Möglichkeit REST-Calls auszuführen.
Es können Bedingungen für die Durchführung eines Steps definiert werden.
So ist es beispielsweise möglich, einen Step nur durchzuführen, wenn eine bestimmte Variable einen bestimmten Wert besitzt.
Es wird ein XML Schema verwendet um sicherzustellen, dass in dem XML-Skript zur Laufzeit keine syntaktischen Fehler vorhanden sind.
\cite{Rotthove.2018}

Ein Nachteil von Workflows ist, dass diese statisch sind, das heißt, dass die Variablenzuweisungen immer zum Zeitpunkt der Erstellung stattfinden.
Dadurch ergibt sich, dass für jede kleine Änderung ein eigener Workflow erzeugt werden muss.
Somit ist ein Workflow eher ein Einmal- bzw. Wegwerfprodukt.

\subsubsection{Template}
Bei dem Nachteil von Workflows als Wegwerfprodukt setzen die sogenannten Templates an.
Ein Template besteht aus drei Dateien.

\begin{enumerate}
\item Datei für Eingabevariablen.\\
In dieser Datei kann Workflowvariablen Werte zugewiesen werden.
Diese Variablen müssen bei ihrer Definition im Workflow entsprechend gekennzeichnet sein.

\item Die sogenannte Aktion-Definitions-Datei.\\
Hier werden die Aktionen, die ein Anwender mit diesem Template durchführen kann, festgelegt.
Einer Aktion wird eine Workflow Definitions Datei und somit ein Workflow zugewiesen.
Darin ist die Datei für die Eingabevariablen und die Festlegung, welche Variablen davon verwendet werden, anzugeben.

\item Die Manifest-Datei.\\
In dieser wird dem Template mitgeteilt, an welchem Speicherort sich die oben genannten Dateien befinden.
Da ein Template immer provisioniert werden kann, wird hier auch der Speicherort des Bereitstellungsworkflows angegeben.
Zusätzlich kann noch eine Beschreibung des Templates hinzugefügt werden.
\end{enumerate}

Somit bildet ein Template einen Rahmen um mehrere Workflows und ermöglicht eine schnellere De-/provisionierung.
Das hat den Vorteil, dass Variablen nur an einer Stelle geändert werden.
Darüber hinaus besteht die Möglichkeit, per Anwendereingabe den Variablen zum Zeitpunkt der Provisionierung einen Wert zuzuweisen.
Somit ist ein Template deutlich flexibler als ein Workflow.
\cite{IBM.2019}

\subsubsection{Instance}\label{sssec:instance}
Hierbei handelt es ich um das Ergebnis nach der Provisionierung eines Templates, zum Beispiel um ein funktionsfähiges CICS.
Als Abgrenzung zu einer Instanz eines Subsystems, kann eine Template Instance auch Instanzen mehrere Subsysteme enthalten.

\subsection{z/OS Provisioning Toolkit}\label{sssec:zospt}
z/OSPT bietet ein Kommandozeileninterface für die Bereitstellung und das Verwalten von Laufzeitumgebungen.
In Abbildung \ref{fig:zospt_help} werden die möglichen Kommandozeilenbefehle mittels des Befehls \glqq zospt -h\grqq{} in einem Kommandofenster angezeigt.
\begin{figure}[h]
	\centering
	\includegraphics[width=\textwidth]{figures/zospt_help_putty.png}
	\caption{z/OSPT mögliche Kommandozeilenbefehle}
	\label{fig:zospt_help}
\end{figure}
Mit z/OSPT werden noch zwei weitere Begriffe eingeführt.\\
\glqq Images\grqq{}.
Dabei handelt es sich grundsätzlich um ein Template, jedoch kann dieses Template über eine weitere Inputdatei verändert werden.
Dadurch kann ein Template mit spezifischen Änderungen provisioniert werden, ohne dass ein neues Template erzeugt werden muss.
Dies erhöht die Flexibilität der Templates weiter.

\glqq Container\grqq{}.
Dabei handelt es sich eins zu eins um eine \glqq Instance\grqq \footnote{Beschreibung in Absatz \ref{sssec:instance}}.
\cite{IBM.2019b}

\subsection{z/OS Management Facility}\label{sssec:zosmf}
Der Hauptaugenmerk dieser Arbeit liegt  bei z/OSMF, da damit die Verwaltung von Workflows und Templates über eine browserbasierende Schnittstelle möglich ist.
Durch diese Oberfläche, in Abbildung \ref{fig:zosmf_welcome} dargestellt, ist es einfacher zu bedienen als das Kommandozeileninterface von z/OSPT  und somit wird der Einstieg in die Provisionierung erleichtert.

\begin{figure}[h]
\centering
\includegraphics[width=\textwidth]{figures/zosmf.png}
\caption{z/OSMF Willkomens Ansicht}
\label{fig:zosmf_welcome}
\end{figure}

Die linke Seite der Abbildung \ref{fig:zosmf_welcome} zeigt den Umfang der z/OSMF  Funktionen.
Für diese Arbeit besitzt nur der Menüpunkt \glqq Cloud Provisioning\grqq{} Relevanz.
Unter diesem Punkt sind die Funktionalitäten für die automatisierte Bereitstellung von Templates zu finden.
\cite{Rotthove.2018}

Zuerst ist das \glqq Resource Management\grqq{} zu nennen..
Darunter werden sogenannte \glqq Domains\grqq{} und die dazugehörigen \glqq Tenants\grqq{} verwaltet.
Unter einer \glqq Domain\grqq{} ist ein System zu verstehen, dass Systemressourcen in Ressourcenpools gliedert.
\glqq Tenants\grqq{} sind die dazugehörigen Rechtegruppen, die dem Anwender den Zugriff auf und die Nutzung von zugeordneten Templates ermöglicht.
Einem Template muss sowohl eine \glqq Domain\grqq{} als auch ein \glqq Tenant\grqq{} zugewiesen werden.
\cite{Rotthove.2018}

Zur Verwaltung der Templates und Instances kommen die \glqq Software Services\grqq{} zum Einsatz.
Dort können neue Templates über die \glqq Manifest Datei\grqq{} hinzugefügt werden.
Dann muss, wie oben beschrieben, eine \glqq Domain\grqq{} und ein \glqq Tenant\grqq{} zugwiesen werden.
Anschließend kann das Template, falls es keine Fehler beinhaltet, veröffentlicht werden.
Es ist zu empfehlen vorher einen \glqq Test Run\grqq{} durchzuführen.
Dabei wird eine Instance testweise provisioniert.
Diese Instance verhält sich genauso wie eine Instance, die aus einem veröffentlichten Template erzeugt wurde.
Somit können damit das Template und die in der Aktion-Definitions-Datei definierten Aktionen getestet werden.
\cite{Rotthove.2018}
