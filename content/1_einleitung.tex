\chapter{Einleitung}\label{ch:einleitung}
\glqq I recently predicted the last mainframe will be unplugged on March 15, 1996\grqq \footnote{\cite{Alsop.1993}}, ein in der Großrechner-Welt bekannt gewordenes Zitat.\
Es handelt sich um eine 1993 getroffene Vorhersage, nämlich dass der letzte Mainframe, auch Großrechner genannt, am 15 März 1996 abgeschaltet werden wird.
Wieso wird sich im Jahre 2020 dennoch mit dieser Technologie beschäftigt?
Bevor darauf eingegangen wird, wird erläutert was ein Großrechner ist.
In einem Satz ist ein Großrechner ein leistungsstarkes, zentralisiertes Serversystem.
Eine genauere Beschreibung ist im Absatz \ref{sec:mainframe} zu finden.
In dieser Arbeit wird nur auf Mainframes aus dem Hause von IBM eingegangen.
Damit wird sich auf einen Technologiestack festgelegt.
Das verwendete Betriebssystem ist z/OS, als Anwendungsserver steht CICS\footnote{Beschreibung im Absatz \ref{cics} zu finden} und als Datenbanksystem Db2\footnote{eine relationale Datenbank} zur Verfügung.
Die Messaging Lösung ist ebenfalls aus dem Hause IBM, dabei handelt es sich um MQ\footnote{Beschreibung im Absatz \ref{sec:mq} zu finden}.
Um nur einige zu nennen.
Unter anderem werden Programmiersprachen wie COBOL, IBM Assembler, C und seit einiger Zeit auch Java verwendet.

Diese Technologie hat auch eine lange Geschichte.
So wurde vor mehr als fünfzig Jahren der allererste Großrechner vorgestellt.
Bis in die 90iger Jahre war dieser unangefochten, jedoch verlor er zu dieser Zeit durch die verteilten Client-Server-Umgebungen an Bedeutung.
Im Jahre 2020 ist allerdings die Cloud sein größter Konkurrent.
Mittlerweile wird der Mainframe als \glqq  legacy\grqq{}- veraltet - betrachtet.
Wieso also wird sich mit dieser Technologie noch beschäftigt? - Als eine Antwort: Es wird weltweit genutzt.
So verarbeiten heutzutage Großrechner weltweit circa 1,2 Millionen CICS Transaktionen pro Sekunde.\footnote{\cite{IBM.2019}}
Im Vergleich hierzu werden 63.000 Google Suchanfragen pro Sekunde abgesetzt. \footnote{\cite{Sullivan.2016}}

Die Kombination aus hohen Workload, dem speziellen Technologiestack und dem Ruf eines veralteten Systems entspringen Risiken.
Es wird immer schwieriger Nachwuchs in diesem Bereich zu finden.
Zum einem da es kaum noch an Universitäten gelehrt wird.
Die Seite des Hochschulkomasses\footnote{\cite{internetagenturKolnFrankfurtsunzinetTYPO3Programmmierung.}} liefert sogar weder für `Mainframe` noch für `Großrechner` keine Treffer.
Zum anderen werden die momentanen Wissensträger älter und scheiden nach und nach aus.
Außerdem holt sich eine Firma, die sich einen IBM Großrechner mit z/OS anschafft oder diesen betreibt, den oben genannten speziellen Technologiestack automatisch mit in die Firma.
Mangels guter alternativen folgt eine starke Herstellerabhänigkeit.

Offentsichlich betreiben dennoch einige Firmen einen IBM Großrechner.
Darunter zählen unter anderem Banken, das Gesundheitswesen, Versicherungen, Fluggesellschaften usw.
Der gemeinsame Nenner dieser Unternehmen ist eine Massendatenverarbeitung, ein hoher Sicherheitsstandard, die dazugehörige Ressourcenverwaltung und Breitband Kommunikation.
All diese Punkte sprechen für die Nutzung eines Großrechners, auch bei der DATEV eG.
\cite{IBM.2014}

Die DATEV eG wurde am 14.02.1966 von 65 Steuerbevollmächtigten gegründet.
Sie verfolgten mit der Gründung das Ziel Buchführungsaufgaben mit Hilfe der EDV zu bewältigen.
Aufgrund hohen Mitgliederwachstums wurde hierfür 1969 in einen firmeneigenen IBM-Großrechner investiert.\cite{DATEVeG.2017}
Heute umfasst das Leistungsspektrum der DATEV eG unter anderen das Rechnungswesen, Personalwirtschaft, Consulting, IT-Sicherheit, Weiterbildung.
Ein nicht unbeträchtlicher Teil, der betriebswirtschaftlichen Anwendungen laufen bis heute auf einem IBM Großrechner.
So werden pro Tag circa 150.000 Batch Jobs\footnote{Beschreibung in Absatz \ref{ssec:job}} und circa 90 Millionen CICS-Transaktionen verarbeitet.
Diese Last wird von circa 14.000 aktiven Modulen erzeugt.
Wie in der Abbildung \ref{fig:Programmiersprachen} zu sehen ist, ist COBOL mit circa 46\% Prozent die am häufigsten verwendete Programmiersprache am Großrechner bei der DATEV eG.
Durch diese Module werden unteranderem im Monat circa 13 Millionen Lohnabrechnungen erstellt und circa 1 Millionen Umsatzsteuer-Voranmeldungen durchgeführt.

Ich persönlich habe eine Ausbildung zum Fachinformatiker der Anwendungsentwicklung im Großrechnerbereich bei der DATEV eG abgeschlossen und parallel dazu ein Studium der Informatik begonnen.
Während dieser Ausbildung lernte ich unter anderem die Entwicklungsprozesse in diesem Bereich kennen.
Kurz bevor ich die Ausbildung begonnen hatte, wurde eine auf eclipse basierende Entwicklungsumgebung für COBOL und IBM Assembler in der DATEV eG eingeführt.
Zuvor wurde mit Hilfe der in Abbildung \ref{fig:3270} gezeigten Oberfläche, die nur ein Syntaxhighlighting zur Verfügung stellt, gearbeitet.
Kurz vor Abschluss meiner Ausbildung wurde git auch für COBOL und IBM Assembler Sourcen als Sourceverwaltung eingeführt.
Dadurch wurde die Sourceverwaltung durch Dateistrukturen oder durch ein von DATEV eG selbstenwickeltes Tool abgelöst.
Das Tooling wurde zwar an einen modernen Entwicklungsprozess angepasst, jedoch nicht der Prozess selbst.
Es wird in der sogenannten \glqq Entwicklungsstage \grqq{} \footnote{Beschreibung der Stages in Absatz \ref{sec:sysplex}} gearbeitet.
So teilen sich zumindest die Entwickler einer Anwendung die gleiche Test-CICS/Db2/MQ Ressourcen.
Dadurch ist eine Absprache bezüglich der Tests, teilweise auch abteilungsübergreifend, unvermeidlich.
Ein weiterer Aspekt eines modernen Entwicklungsprozesses wäre, die automatisierte Bereitstellung von Systemumgebungen, zum Beipiel einer Datenbank.
Hierfür muss im aktuellen Mainframeentwicklungsprozess ein manueller Ablauf mit mehreren beteiligten Abteilungen durchlaufen werden.
Dadurch wird dieser fehleranfällig und langsam.

Im Vergleich hierzu durfte ich nach dem Abschluss meiner Ausbildung einen modernen Entwicklungsprozess in einem \glqq Cloud Native \grqq-Projekt selbst erleben.
So hatte jeder Entwickler seine eigene Testumgebung und konnte ungestört von seinen Kollegen, ein neues Feature testen.
Die Datenbank konnte per Knopfdruck ersetzt werden, ganze Laufzeitumgebungen konnten so erzeugt werden.
Hier kam die Frage auf, ob es möglich wäre diese Prozesse auch für legacy z/OS Anwendungen einsetzen zu können.
In Zusammenarbeit mit dem Bereich der Technologiestrategie ergab sich das Thema dieser Arbeit.

Ist es generell technisch möglich mit dem \glqq IBM Cloud Provisioning and Management for z/OS\grqq-Toolkit Laufzeitumgebungen automatisiert bereitzustellen?
Wird dadurch der aktuelle Bereitstellungsprozess schneller und auch sicherer?
Erzeugt die Nutzung einen Mehrwert bei den Stakeholdern, also den Entwicklerteams und den Administratorenteams.

Um diese Fragen zu beantworten wird die Provisionierung einer Laufzeitumgebung anhand einer speziellen Anwendung untersucht.
Die Anwendung sollte ein CICS als Anwendungsserver, eine Db2 Datenbank und MQ als Messaginglösung benötigen.
Um für alle drei Breiche eine Aussage treffen zu können.
Die genaue Vorgehensweise wird im Kapitel \ref{ch:vorgehensweise} beschrieben.