\chapter{Einleitung}\label{ch:einleitung}
Vor mehr als fünfzig Jahren wurde der allererste Großrechner, auch Mainframe genannt, vorgestellt.
Seit dieser Zeit setzen sich die monolitisch aufgebauten Systemes im Bezug auf Leistungsfähigkeit und Zuverlässigkeit gegenüber andere Systeme ab.
Obwohl die Systeme immer weniger Platz brauchten, anfangs waren es ganze Gebäudestockwerke, heute sind es ungefähr die Ausmaße eines großen Kleiderschrankes.
Und weiteren Verbesserung bei der Handhabbarkeit, von reinen Druckausgaben über text-basierenden Terminals bis hin zu benutzerfreundlichen GUI´s.
Auch hat sich die Weise, wie Programme entwickelt werden verändert.
Zu Beginn mussten diese noch auf Lochkarten (ABBILDUNG !!) gestanzt und umständlich über ein Lesegerät eingelesen werden.
Heute stehen dem Entwickler moderne IDE´s zur Verfügung.
Trotz dieser Veränderungen verlor der Mainframe durch die Dezentralisierung der IT hinzu Client-Server-Umgebungen in den 1990-er Jahren an Bedeutung.
\cite{Ceruzzi.2003}

Seit den 1990-er 
Mainframe Deatch \cite{StewartAlsop.1993}

Was ist der Mainframe\\
Geschichte\\
kurzer technischer Einblick Mainframe und Vorteile und (Bedeutung bei Datev) \\
wieso er bei Datev eingesetzt wird \\
 Moma \\
 Problemstellung \\
 Wieso Rechnungsschreibung als Beispiel? \\