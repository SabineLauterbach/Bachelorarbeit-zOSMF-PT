\chapter{Zusammenfassung}\label{ch:zusammenfassung}
Zusammenfassend lässt sich sagen, dass es generell möglich ist mit dem \glqq IBM Cloud Provisioning and Management for z/OS\grqq-Toolkit Laufzeitumgebungen für legacy z/OS Anwendungen automatisiert bereitzustellen.
Das funktionsfähige Template verkürzt den Bereitstellungsprozess deutlich.
Durch den Abbau der Kommunikation zwischen den Abteilungen und nur einmaligem Erstellen der Skripte ist es zudem weniger fehleranfällig.
Die Stakeholder sehen in diesem Template auch einen Mehrwert.
Jedoch ist es noch nicht perfekt.
Der Bereitstellungsprozess ist noch immer mit einigen manuellen Schritten verbunden.
So muss das Template manuell kopiert werden und Änderungen an der Konfiguration müssen innerhalb des Templates stattfinden.

Hierfür wurde in der Arbeit eine Lösung mit Hilfe von z/OSPT beleuchtet.
Diese sieht in einer Endausbaustufe eine einfache Einbindung des Templates in einen automatisierten Build-Prozess, zum Beipsiel mit Jenkins, vor.
Außerdem würde der Einsatz von z/OSPT das Einbinden in den DATEV eG internen \glqq Marktplatz\grqq{} für Cloud Lösungen ermöglichen.
Um diese Ziele zu erreichen muss noch viel Aufwand in die Gestaltung des Templates gesteckt werden.
Zusätzlich müsste ein sogenannter \glqq Service Broker\grqq{} für die Einbindung der einzelnen Subsysteme in den \glqq Marktplatz\grqq{} implementiert werden.
Diese beiden Lösungsansätze stoßen sowohl bei den Administratorenteams als auch beim involvierten Entwicklerteam auf fruchtbaren Boden.
Dadurch wird eine Ähnlichkeit zum Cloud Native-Bereitstellungsprozesses hergestellt.
Dies ist ein weiterer Schritt um dem Image eines veralteten Systems mit veralteten langsamen Prozesses zu entkommen.  
