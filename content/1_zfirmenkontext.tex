\chapter{Motivation der DATEV e.G.}\label{ch:Firmenkontext}
Die in der Einleitung genannten Fragen werden aktuell im Rahmen des Querschnittsprojekts \glqq Mainframestrategie\grqq{} innerhalb der DATEV e.G. diskutiert. 
Der DATEV e.G. sind die zunehmenden Risiken durch schwierige Bereitstellung von Skills und der Herstellerabhängigkeit bewusst. 
Die Frage ist, wie wird mit den vielen Mainframebestandsanwendungen in Zukunft umgegangen?


Diese Arbeit beschäftigt sich im Schwerpunkt mit den Aspekten der Entwicklungs- und Betriebseffizienz. 
Ziel ist es, diese zu steigern und eine Homogenisierung von Skills zwischen Mainframe-Entwicklern und Cloud-Entwicklern zu erreichen. 
Zur Entwicklungseffizienz gehört z.B. die in der Einleitung genannte moderne, auf eclipse basierende Entwicklungsumgebung ID/z (Kommentar, irgendwie verlinken auf Glossar o.ä.).
Diese wird bei DATEV e.G. schon seit ca. 10 Jahren firmenweit bereitgestellt.
Relativ neu (Seit 2018) ist die Nutzung von GIT als Sourceverwaltung für z/OS Artefakte. . (KOmmentar: hier könnten wir ein Zitat von mir einfügen, dass ich über Konferenzteilnahmen andere Kunden kenne und deshalb weiß, wie schwer sich Mainframe-Kunden mit GIT tun). 





