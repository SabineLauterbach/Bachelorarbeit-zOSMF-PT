\chapter{Firmenkontext}\label{ch:Firmenkontext}
Die Informationen stammen aus Gesprächen mit einem Mitarbeiter aus dem Technologiestrategieteam und der im Anhang \ref{app:momappp} befindlichen Präsentation.
Die in der Einleitung genannten Fragen passen zur momentanen Mainframestrategiediskussion innerhalb der DATEV eG.
Der DATEV eG sind die zunehmenden Risiken, durch schwierigere Bereitstellung von Skills und der Herstellerabhänigkeit, bewusst.
Die Frage ist, wie wird mit den vielen Mainframebestandsanwendungen in Zukunft umgegangen.
Die komplette Ablösung dieser Anwendungen durch cloud-native Lösungen ist aktuell nicht absehbar.
Die Funktionsfähigkeit dieses Bestandsgeschäfts, welches eine große Einnahmequelle der DATEV eG ist, muss durch effiziente Weiterentwicklung und Wartung gewehrleistet werden.
Auch im Falle einer Ablöse muss je nach Strategie das Alt-System weiterhin über Jahre oder Jahrzente gepflegt werden.
Daraus folgt, dass aus Sicht der DATEV eG keine Investition in die IBM Mainframe Plattform keine Alternative darstellt.

In diesem Umfeld wurde die in der Einleitung bereits genannte Umstellung auf eine moderne auf eclipse basierende Entwicklungsumgebung und auf git als Sourceverwaltung umgesetzt.
Des Weiterene läuft eine Untersuchung bezüglich automatisierter Builds mit Hilfe von Jenkins basierten Pipelines.
Diese Maßnamen erleichtern Mainframe fremden Entwicklern den Einstieg in diese spezielle Welt.
Dennoch ist der Prozess für die Bereitstellung von Middleware noch unverändert und noch nicht modernisiert.
Hier setzt diese Arbeit an und klärt die in der Einleitung genannten Fragen.


