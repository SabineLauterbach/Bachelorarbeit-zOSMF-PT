\chapter{Analyse}\label{ch:analyse}
Im Folgendem erfolgt eine Beschreibung der Beispielanwendung `Rechnungsschreibung`.
Hierbei wird vor allem der technische Aspekt beleuchtet.
Anschließend wird der aktuelle Bereiststellungsprozess für Laufzeitumgebungen, den dazugehörigen Datenbanksystem und einer Messaging Lösung dargestellt.

\section{Beschreibung Rechnungsschreibung}
Die DATEV eG bietet ihren Kunden kostenpflichtige Leistungen an, wenn ein Kunde eine solche in Anspruch nimmt ist die DATEV eG nach § 14 Absatz 2 UStG verpflichtet eine Rechnung auszustellen.
Bei einer Rechnung handelt es sich nach § 14 Absatz 1 UStG um ein Dokument, mit dem über eine Lieferung oder sonstige Leistung abgerechnet wird.

Die Erzeugung der Rechnungen lässt sich in mehrere Schritte unterteilen, gesammelt werden diese Schritte als `Rechnungsschreibung' bezeichnet.\\
Zunächst wird nach jeder kostenpflichtigen Leistungserbringung durch die dazugehörige Anwendung ein Berechnungssatz erzeugt.
Ein Berechnungssatz beinhaltet die Metainformationen der Berechnung unter anderem die Artikelnummer, Menge und den Ordnungsbegriff.
Der Preis und der Rechnungsempfänger wird zu einem späteren Zeitpunkt innerhalb der `Rechnungsschreibung' ermittelt.
So fallen in der DATEV eG monatlich circa 30 Millionen Datensätze an.
Das Einpflegen der Berechnungssätze in den Rechnungsschreibungsablauf erfolgt über die allgemeine DMVINF\footnote{DatevMakroVerarbeitungsinformation}-Schnittstelle.
Für die Schnittstelle wurde die Programmiersprache Assembler verwendet.

Die Anwendung hat drei Möglichkeiten die DMVINF-Schnittstelle zu bedienen. \\
Bei der Ersten Möglichkeit handelt es sich um das direkte Befüllen der Schnittstelle.
Während des gesamten Übergabeprozesses dürfen von der sendenden Anwendung nur bestimmte Informationsfelder (Ordnungsbegriffe, Länderschlüssel und Mengen) geändert werden.
Zusätzlich werden alle variablen Informationen einer Formalprüfung unterzogen.
Außerdem wird ein Kontrollsatz mit Anzahl der Sätze und Summe der einzelnen Artikel in den Berechnungsdaten hinterlegt.
Diese Massnahmen sorgen dafür, dass auch im späteren Verlauf Datenmanipulation ausgeschlossen ist. \\
Eine weitere Möglichkeit die DMVINF-Schnittstelle zu bedienen ist, die Übergabe über einen WebService.



\section{Aktueller Bereitstellungsprozess}
Mit vielen anderen Abteilungen sprechen\\
Viel auf `Zuruf` und Besprechungen\\
Genauere Infos noch von den CICSAdmins nachfragen\\

 