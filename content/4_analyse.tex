\chapter{Analyse}\label{ch:analyse}
Im Folgendem wird der aktuelle Bereitstellungsprozess für die Laufzeitumgebung, dem dazugehörigen Datenbanksystem und einer Messaging Lösung getrennt voneinander dargestellt.
Anschließend erfolgt eine Beschreibung der Beispielanwendung \glqq DATEV-Rechnungsschreibung\grqq.
Die dafür benötigten Informationen stammen aus Gesprächen mit Mitarbeiter 1 aus der Abteilung, die für die DATEV-Rechnungsschreibung zuständig ist.
Es wird vor allem der technische Aspekt beleuchtet.

\section{Aktueller Bereitstellungsprozess}\label{sec:aktbereit}
Die in diesem Absatz genannten Informationen stammen aus Gesprächen mit Mitarbeitern aus den jeweiligen Administratorenteams.
Wie in Absatz \ref{sec:zosanw} beschrieben benötigt eine z/OS Anwendung zunächst eine Laufzeitumgebung, im Fall dieser Arbeit handelt es sich um CICS.

\subsection{Bereitstellung einer CICS-Instanz}
Um eine lauffähige CICS-Instanz einzurichten, sind mehrere Schritte notwendig.
Der komplette Prozess wird in Abbildung \ref{fig:aktcics} dargestellt.
Wie zu sehen ist, ist der Initiator des Prozesses das Entwicklerteam.
Zunächst wird dort während der Entwicklungsphase festgestellt, dass ein neue CICS-Instanz benötigt wird.
Hier hilft das CICS Administratorenteam mittels Beratung aus.
Während einer Beratungsphase, die via Telefon, Emails oder Terminen stattfindet, wird sichergestellt, ob wirklich eine neue CICS-Instanz notwendig ist oder ob nicht eine bereits bestehende Instanz genutzt werden kann.
Falls eine neue CICS-Instanz benötigt wird, wird ein RACF Eintrag für diese Instanz beantragt.
Dieser Eintrag wird dann vom RACF Team erzeugt.
Um sicherzustellen, dass die CICS-Instanz in den täglichen Sicherungen enthalten ist, muss das System Automations-Team benachrichtigt werden.

\begin{figure}[ht!]
\centering
\includegraphics[width=\paperwidth,angle=90]{figures/swimlaneCICS.pdf}
\caption{Bereitstellungsprozess einer CICS-Instanz}
\label{fig:aktcics}
\end{figure}

Nun kann mit dem eigentlichen Anlegen der CICS-Instanz begonnen werdem.
Dabei müssen folgende Schritte manuell durchgeführt werden.
Es werden nur die Schritte, die im Laufe dieser Arbeit durch das \glqq IBM Cloud Provisioning and Management for z/OS\grqq-Toolkit automatisiert werden, dargestellt.
Es handelt sich um das Anlegen von:

\begin{itemize}
\item CICS spezifische Dateien
\item \glqq CICS System Definition\grqq
\item Started Task Control-Job
\end{itemize}

\paragraph{CICS spezifische Dateien}\label{sssec:speziDat} ~\\
Zunächst müssen CICS spezifische Dateien im z/OS angelegt werden.
Im Fall des dieser Arbeit zugrunde liegenden Beispiels handelt es sich um siebzehn verschiedene VSAM\footnote{Virtual Storage Access Method, spezielle Dateiart, die schnelle I/O-Zugriffe ermöglicht.\cite{Lovelace.2013}} Dateien.
Diese Dateien benötigt die CICS-Instanz um zum Beispiel Systemfehler zu protokollieren oder den Debugger aktivieren zu können.

\paragraph{\glqq CICS System Definition\grqq} ~\\
In der Datei \glqq CICS System Definition\grqq, kurz CSD, muss jede Ressource, die dem System zur Verfügung stehen soll, definiert werden.
Eine CSD Datei kann für mehrere CICS-Instanzen verwendet werden und besteht aus mehreren Einträgen.
Ein Eintrag besteht aus einer Gruppe und einer Liste.
Die Gruppe ist hierbei die Definition einer Systemressource und muss manuell angelegt werden.
Bei der Liste handelt es sich um das System, welches diese Ressource benötigt.
Dort ist unter anderem für jede CICS-Instanz hinterlegt, zu welchem Db2 Datenbanksystem und welchem IBM MQ Messagingsystem sich diese Instanz verbinden soll.

\paragraph{Started Task Control-Job} ~\\
Bei einem Started Task Control-Job, kurz STC Job, handelt es sich um einen Batch Job, der mit Hilfe des \glqq START\grqq-Konsolenkommandos innerhalb von z/OS gestartet werden kann.
Dieser Batch Job wird deshalb auch als Started Task bezeichnet.\cite{Cassier.2007}
Bei der DATEV e.G. existiert für jede Instanz eines Subsystems ein solcher Job, so also auch für CICS.
In diesem werden zunächst einige zur Laufzeit benötigten Bibliotheken und Dateien eingebunden, unter anderem die CICS spezifischen Dateien\footnote{Beschreibung in Absatz \ref{sssec:speziDat}}.
Außerdem werden hier die SIT \footnote{CICS system initialization table} Parameter definiert.
Zunächst wird festgelegt welche Standard SIT verwendet werden soll.
Anschließend können diese Standardwerte überschrieben werden.
Zu diesen Parametern zählen unter anderem der eindeutige Name der CICS-Instanz, der Speicherort der dazugehörenden CSD und die Information, ob eine Verbindung zu einem Db2 Datenbanksystem hergestellt werden soll.

Nach der Durchführung dieser Schritte und einem erfolgreichen Startversuch, steht dem Entwickler eine neue CICS-Instanz zur Verfügung.
Der komplette Ablauf dauert, unter der Annahme, dass alle Beteiligten verfügbar sind, nur diese Anforderung umsetzen müssen und für die Beratung ein Arbeitstag veranschlagt wird, circa zwei Arbeitstage.

\subsection{Bereitstellungsprozess einer Db2 Datenbank}
Wie in Abbildung \ref{fig:aktdb2} zu erkennen ist, ist der Bereitstellungsprozess einer neuen Db2 Datenbank mit vielen Aufgaben im Entwicklerteam verbunden.
\begin{figure}[ht!]
\centering
\includegraphics[width=\paperwidth,angle=90]{figures/swimlaneDb2.pdf}
\caption{Bereistellungsprozess einer Db2 Datenbank}
\label{fig:aktdb2}
\end{figure}
Zunächst müssen sogenannte Projektinformationen, unter anderem Daten der Voruntersuchung, vom Entwicklerteam bereitgestellt werden.
Das Projektkürzel, der Datenbank- und Projektname und die Projektbezeichnung müssen mit den involvierten Abteilungen besprochen werden.
Über den sogenannten \glqq Datenbankänderungsantrag\grqq{} wird ein Genehmigungsprozess angestoßen.
Wenn alle Genehmigungen erteilt wurden, kann ein Dateneigentümer festgelegt werden.
Anschließend muss die Datenbank mittels eines Datenbankmodells vom Entwicklerteam beschrieben werden und eine Usergruppe, die im späteren Verlauf die Datenbankzugriffsrechte benötigt, beantragt und angelegt werden.
Die eigentliche manuelle Erstellung der Datenbank wird mittels des Datenbankmodells und den Projektinformationen im Anschluss dazu durchgeführt.

Die Zugriffsrechte für die zuvor beantrage Usergruppe auf die neue Datenbank werden beantragt.
Schließlich steht dem Entwicklerteam die neue Db2 Datenbank zur Verfügung.
Wird die Db2 Datenbank in Verbindung mit einem CICS verwendet, so sind weitere manuelle Schritte vom CICS Administratorenteam notwendig.
Der komplette Ablauf dauert, unter der Annahme, dass alle Beteiligten verfügbar sind, nur diese Anforderung umsetzen müssen und für die Beratung ein Arbeitstag veranschlagt wird, circa zwei Arbeitstage.

\subsection{Bereitstellungsprozess einer IBM MQ Queue}
Auch bei dem Bereitstellungsprozess, siehe Abbildung \ref{fig:aktmq}, einer IBM MQ Queue ist das Entwicklerteam der Initiator.

\begin{figure}[ht!]
\centering
\includegraphics[width=\paperwidth,angle=90]{figures/swimlaneMQ.pdf}
\caption{Bereistellungsprozess einer IBM MQ Queue}
\label{fig:aktmq}
\end{figure}

Die Grundlage diese Prozesses ist ein Antrag auf Erstellung einer neuen IBM MQ Queue.
Zuvor findet eine Beratung via Telefon, Email oder Terminen statt.
Die Queues werden anschließend manuell eingerichtet und stehen dem Entwicklerteam zur Verfügung.
Wird die Queue in Verbindung mit einem CICS verwendet, so sind weitere manuelle Schritte vom CICS Administratorenteam notwendig.
Trotz des scheinbar schmalen Prozesses dauert der Ablauf unter der Annahme, dass alle Beteiligten verfügbar sind, nur diese Anforderung umsetzen müssen und für die Beratung ein Arbeitstag veranschlagt wird, circa zwei Arbeitstage.

\subsection{Zusammenfassung aktueller Bereitstellungsprozess}
Wie in den drei Diagrammen, Abbildungen \ref{fig:aktcics}, \ref{fig:aktdb2} und \ref{fig:aktmq}, zu erkennen ist, ist der aktuelle Bereitstellungsprozess noch mit vielen manuellen Schritten verbunden.
Außerdem ist der Hauptaufwand in den Administratorenteams angesiedelt.
Das Entwicklerteam ist der Initiator des Ablaufs.
Folglich kümmert es sich um Formulare und die erste Kontaktaufnahme zum Administratorenteam.

Zusätzlich zu den vielen manuellen Schritten sind die vielen Absprachen zwischen mehreren Abteilungen zu nennen.
Steht ein beteiligtes Team nicht zu Verfügung, kommt es zu Verzögerungen, das Team muss warten, der komplette Zeitplan kann sich dadurch nach hinten verschieben.
Der Prozess für die Bereitstellung einer CICS-Instanz, mit einer Db2 Datenbank und IBM MQ Queues dauert in der Summe circa sechs Arbeitstage.
Es setzt sich aus der Dauer der Einzelprozesse zusammen, für jedes Subsystem wird mit circa zwei Arbeitstagen gerechnet.
Natürlich ist ein parallelisierter Ablauf der einzelnen Teilprozesse möglich, so kann die Gesamtdauer im besten Fall auf circa zwei bis drei Arbeitstage verkürzt werden.

Ein weiterer Punkt ist, dass die Kommunikation beziehungsweise der Initiator für den Start des gesamten Prozesses meist per Zuruf stattfindet.
So existiert für die erste Kontaktaufnahme kein Formular, keine Automation oder ähnliches.
Zur Kommunikation wird auf E-Mail, Telefon oder mittels Terminen zurückgegriffen.

\section{DATEV-Rechnungsschreibung}\label{rechBesch}
Für diese Arbeit wurde die DATEV-Rechnungsschreibung als Beispielanwendung herangezogen, weil sie folgenden Anforderungen entspricht.
Es handelt sich zum einem um eine in sich abgeschlossene Anwendung, die nur zu Beginn des Prozesses von anderen Anwendungen abhängig ist.
Zum anderen benötigt die DATEV-Rechnungsschreibung ein CICS als Laufzeitumgebung, eine Db2-Datenbank und IBM MQ als Messaginglösung.
Somit kann ein umfangreicher Bereitstellungsmechanismus untersucht werden.

Bei dem Gesamtablauf handelt es sich um einen Batch-Ablauf auf dem Großrechner der DATEV e.G.
Dieser setzt sich aus folgenden Teilen zusammen:
\begin{itemize}
\item Sammeln von Berechnungssätze
\item Tägliche Bewertung
\item Rechnungsaufbereitung
\end{itemize}
Für die Beantwortung der Forschungsfragen ist nur ein Teil der \glqq Tägliche Bewertung\grqq relevant, die Preisermittlung.

\subsection{Tägliche Bewertung}\label{sssec:täglbew}
Dieser Ablauf läuft einmal täglich von Montag bis Freitag und ist für die Preisermittlung und Kundenzuordnung zuständig.
Zur Realisierung wurden die Programmiersprachen Assembler, COBOL und Java genutzt.
Am Ende dieses Ablaufes steht die ARUBA\footnote{Abrechnungs- und Umsatz-Basis}-Db2-Datenbank.
Dort werden die Berechnungsdaten der letzten 36 Monate aufbewahrt.
Dabei handelt es ich um insgesamt circa 3,8 Milliarden Datensätze von einer Gesamtgröße von circa 400 GB mit Indizes.
Diese Datensätze beinhalten alle Informationen für die endgültige Erzeugung der Rechnungen.

\subsection{Preisermittlung}
Die Preisermittlung ist für die Berechnung der Preise mit den dazugehörigen kundenindividuellen Abhängigkeiten, beispielsweise Rabatte, zuständig.
Die Eingabe beläuft sich an Lasttagen auf bis zu 180.000 Geschäftspartner.
Im DATEV e.G. Umfeld ist ein Geschäftspartner entweder eine Kanzlei oder ein einzelner Mandant.
Aufgrund dieser Last wurde die Berechnung zum einen in CICS-Instanzen ausgelagert und zum anderen wurde der Ablauf in zwei Teile zerlegt:
\begin{itemize}
\item Bereitstellen der Preisinformationen
\item Berechnung der Preise
\end{itemize}
\paragraph{Bereitstellen der Preisinformationen}~\\
Bevor die eigentliche Ermittlung der Preise stattfindet, werden zunächst die Preisinformationen und die kundenindividuellen Preisabhängigkeiten, wie zum Beispiel Rabatte, ermittelt.
Für die Verarbeitung werden zwei Queues verwendet.
Eine startet eine Transaktion im CICS, die andere wartet auf deren Antwort.
Innerhalb der Transaktion werden alle benötigten Preisinformationen und -abhängigkeiten mit Hilfe einer Db2 Datenbank ermittelt.
Diese Informationen werden dann in einem sogenannten \glqq SHARED GETMAIN\grqq-Bereich gespeichert.
Dabei handelt es sich im Prinzip um einen Hauptspeicherbereich, der dem CICS Subsystem zur Verfügung steht.
Die Adresse dieses Bereiches wird den Transaktionen zur Verfügung gestellt.
Somit greifen die einzelnen Transaktionen nicht mehr direkt auf die Datenbank zu, sondern stattdessen auf den schnelleren Hauptspeicher.
Diese Vorarbeit ist notwendig, da es aufgrund von bis zu 60 Millionen Datenbankzugriffen zu massiven Einbußen bezüglich der Performance führen würde.

\paragraph{Berechnung der Preise}~\\
Um die Berechnungsdaten der 180.000 Geschäftspartner an CICS-Instanzen zu übertragen, stehen dem System weitere Queues zur Verfügung.
Darunter ist eine allgemeine Queue in der alle Aufträge, die für die Weiterverarbeitung zur Verfügung stehen, geschrieben werden.
Pro Geschäftspartner wird ein Auftrag angelegt.
In diesem Auftrag befinden sich die Namen vier weiterer Queues.
Eine dieser Queues beinhaltet alle Informationen, die für die Preisermittlung des dazugehörigen Geschäftspartners notwendig sind.
Hierzu zählt unter anderem die Adresse des vorher beschriebenen Hauptspeicherbereichs.
In den restlichen drei Queues sind die Ergebnisse der Preisermittlung gespeichert.
Die Ergebnisse stehen somit dem Batch-Ablauf zur Weiterverarbeitung zur Verfügung.
Für jede der vier Queues existieren jeweils 100 vorgefertigte Namen.
Somit können auch maximal nur 100 Aufträge gleichzeitig auf Weiterverarbeitung warten.
Falls dieses Limit erreicht ist, wartet der Batch-Ablauf so lange, bis einer der Aufträge fertig gestellt wird.
Sobald ein Auftrag in die allgemeinen Auftragsqueue geschrieben wird, wird eine CICS-Transaktion gestartet.
Diese führt die Preisermittlung durch und schreibt das Ergebnis auf die dazugehörigen Queues.
Ist dies geschehen, stehen die Queues wieder für einen neuen Auftrag zur Verfügung.
Es können maximal 30 Transaktionen zeitgleich arbeiten.