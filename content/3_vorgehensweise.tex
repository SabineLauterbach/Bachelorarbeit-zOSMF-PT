\chapter{Vorgehensweise}\label{ch:vorgehensweise}

Einarbeitung in die Thematik
Analyse Ist-Zustand (inkl. Beschreibung der Anwendung)
(evtl. erster Workshop erwähnen)
Tool im Testplex (Testumgebung derAdmins) zuerst nur die Laufzeitumgebung, dann die Anforderungen der Anwendung nach und nach mit einbauen (HIER KEINE DATEN VORHANDEN), wirklich nur die Umgebung 
Tool in der Entwicklungsumgebung (Testumgebung der Entwickler) wie auf dem Testplex versuchen und dann noch die Daten und die eigentliche Anwendung mit einbeziehen
Am Ende Nutzwertanalyse mit Admins und Entwicklern
(Evtl. noch einen Workshop bzw. Vorstellung der Ergebnisse)
