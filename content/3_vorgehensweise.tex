\chapter{Vorgehensweise}\label{ch:vorgehensweise}
Zu Beginn dieser Arbeit war eine Einarbeitung nicht nur in das verwendete Toolkit notwendig.
Sondern es musste sich viel mit den verschiedenen Systemen, also CICS, Db2 und MQ, im Hinblick auf administrative Aufgaben auseinander gesetzt werden.
Nachdem eine Beispielanwendung gefunden wurde, folgte die Analyse des Ist-Zustandes inklusive die Beschreibung dieser Anwendung.
Während der Analyse wurde der momentane Bereitstellungsprozess untersucht.

(evtl. Workshop erwähnen)

Die Installation des Toolkits geschah bereits vor dem Beginn dieser Arbeit.
Somit ist es möglich auf dem Testplex, einer Systemumgebung für Test von neuen Betriebssystemversionen oder ähnlichem, zu beginnen.
Dieser wird hauptsächlich von Administratoren genutzt.
Zusätzlich ist dieser komplett von anderen Systemumgebungen abgekapselt um unvorhersehbare Fehler zu vermeiden.
In dieser Umgebung wird zunächst untersucht, wie es möglich ist ein CICS zu provisionieren.
Hierfür wird vorerst ein von der IBM bei der Installation von z/OSMF mitgeliefertes minimales CICS Template verwendet.
Anschließend wird ein umfangreicheres mitgeliefertes Template an die Anforderungen der Anwendung angepasst.
Da im Testplex jedoch keine Anwendungsdaten vorhanden sind, wird vorerst nur die benötigten Systeme provisioniert und Grundvoraussetzungen geschaffen.

Nachdem eine CICS Instanz, eine Db2 Datenbank und MQ Queues auf dem Testplex sowohl provisioniert als auch deprovisioniert werden können, folgt der nächste Schritt.
Dabei handelt es sich um den Wechsel der Systemumgebung vom Testplex auf die Entwicklungssystemumgebung.
Letzteres ist die Testumgebung für alle Mainframe Entwickler.
Hier werden vor allem neue Programmversionen entwickelt und damit kleinere Tests durchgeführt.
Außerdem sind in der Entwicklungssystemumgebung alle Anwendungsdaten, die für diese Tests notwendig sind vorhanden.
Somit kann die Integrierung der Beispielanwendung in das provisionierte CICS stattfinden.

Zu letzt wird eine Diskussionsrunde stattfinden.
Hierbei werden Kollegen aus allen beteiligten Gruppen teilnehmen.
Dazu zählen CICS-Administratoren, Db2-Administratoren, MQ-Administratoren, Entwickler der Beispielanwendung und Architekten.
Zunächst wird das Ergebnis dieser Arbeit vorgestellt.
An Hand dessen wird diskutiert, ob und wenn ja wie das `IBM Cloud and Management for z/OS Toolkit` verwendet werden soll.