\chapter{Vorgehensweise}\label{ch:vorgehensweise}
Zu Beginn dieser Arbeit wurde zunächst das `IBM Cloud Provisioning and Management for z/OS`-Toolkit mit seinen Varianten zOSMF und zOSPT analysiert.
Anschließend folgt die Analyse und Darstellung des Ist-Zustandes einer Beispielanwendung, inklusive der Beschreibung des momentan etablierten Bereitstellungsprozesses für CICS, Db2 und MQ.
Schließlich wird an Hand einer Beispielimplementierung gezeigt, ob es technisch möglich ist eine Laufzeitumgebung automatisiert mit dem oben genannten Toolkit für diese Beispielanwendung bereitzustellen.

Im folgenden wird die Vorgehensweise während der Beispielimplementierung erläutert.
So wird zunächst auf einer Systemumgebung für Tests von neuen Betirebssystemversionen oder ähnlichem, dem sogenannten Testplex, begonnen.
Dieser ist komplett von anderen Systemumgebungen abgekapselt, um unvorhersehbare Fehler zu vermeiden.
In dieser Umgebung wird zunächst untersucht, wie es möglich ist ein CICS zu provisionieren.
Hierfür wird vorerst ein von der IBM bei der Installation von z/OSMF mitgeliefertes minimales CICS Template verwendet.
Anschließend wird ein umfangreicheres mitgeliefertes Template an die Anforderungen der Anwendung angepasst.
Für die Erstellung von Db2 Datenbanken existiert innerhalb der bereits eine REST-API.
Es wird versucht diese innerhalb des Templates zu nutzen.
MQ Queues werden mittels Standard IBM Jobs provisioniert.
Da im Testplex jedoch keine Anwendungsdaten vorhanden sind, wird vorerst nur die benötigten Systeme provisioniert, um so die Grundvoraussetzungen zu schaffen.

Nachdem eine CICS Instanz, eine Db2 Datenbank und MQ Queues auf dem Testplex sowohl provisioniert als auch deprovisioniert werden können, folgt der nächste Schritt.
Dabei handelt es sich um den Wechsel der Systemumgebung vom Testplex auf die Entwicklungssystemumgebung.
Letzteres ist die Testumgebung für alle Mainframe Entwickler.
Hier werden vor allem neue Programmversionen entwickelt und damit kleinere Tests durchgeführt.
Außerdem sind in der Entwicklungssystemumgebung alle Anwendungsdaten, die für diese Tests notwendig sind, vorhanden.
Somit kann die Integrierung der Beispielanwendung in das provisionierte CICS stattfinden.

Zuletzt werden Interviews mit Vertretern aus dem Administratorenteam, Entwicklerteam und dem Team für die Technologiestrategie, abgehalten.
Dadurch erfolgt eine Bewertung bezüglich des möglichen Nutzwertes und ob die Technologie von den Stakeholdern akzeptiert wird.