\chapter{Realisierung}\label{ch:realisierung}
In diesem Kapitel wird beschrieben, wie die Aufgabe dieser Arbeit gelöst wurde.
Dazu wird nach der im Kapitel \ref{ch:vorgehensweise} beschriebenen Reihenfolge, der Arbeitsschritte vorgegangen.
Es ist nochmal zu erwähnen, dass zunächst die Provisionierung einer CICS Instanz untersucht wird.
Danach wird in weiteren Schritten zuerst eine Db2 Datenbank und schließlich MQ Queues dem Bereitstellungsprozess hinzugefügt.

\section{Testplex}
Vor Beginn der eigentlichen Untersuchung mussten zunächst alle benötigten Rechte beantragt werden.
Hierzu zählen unter anderem die Rechte für die Nutzung des Testplexes, die Nutzung von z/OSMF und z/OSPT und die Rechte für die Templateverwaltung innerhalb von z/OSMF.
Außerdem war es auf dem Testplex möglich, die Rechte für das Erstellen der CICS Dateien, das Recht, um ein CICS starten zu dürfen und die Rechte für die Administration von Db2 und MQ einer persönlichen UserID zu geben.
Dies stellt kein Problem dar, weil es sich bei dem Testplex um eine reine Systemtestumgebung handelt.
Außerdem benötigt das IBM Cloud and Management for z/OS lesenden Zugriff auf den Speicherpfad der Template Dateien.
Schließlich konnte mit dem ersten Versuch ein bei der Installation von z/OSMF mittgeliefertes minimales CICS Template zu provisionieren begonnen werden.

\subsection{IBM Standard CICS Template}
Trotz der Vorteile durch die Weboberfläche von z/OSMF wurde zunächst auf z/OSPT gesetzt.
Diese Entscheidung fiel auf Grund der höheren Flexibilität, durch Images.
Da es sich um ein mitgeliefertes Template handelt, sind alle benötigten Workflow Definitionsdateien und Template Dateien vorhanden.
Somit konnte der Konsolenbefehl `zospt build` auf dieses Template durchgeführt werden.
Dadurch sollte ein Image erzeugt werden.
Jedoch zeigte sich ein weiterer Nachteil des Kommandozeileninterfaces, es ist nicht möglich Templates eine Domain und einen Tenant zuzuweisen.
Dies hatte zur Folge das der Befehl `zospt build` fehlschlug.
Zusätzlich führte es dazu, dass für alle folgenden Aufgaben z/OSMF genutzt wird.

Das z/OSMF auf die Template- und Workflow-Dateien zugreifen kann, sind diese in einem Unix Dateisystem auf dem Großrechner gespeichert.
Das Template konnte dann ohne weitere Probleme in die Software Services aufgenommen werden.
Dabei wurden ihm eine Domain und ein Tenant zugewiesen.
Bevor das Template provisioniert werden kann, müssen Änderungen in der Eingabevariablen Datei vorgenommen werden.
Dazu mussten die Werte, der in Tabelle \ref{tab:cgsvars} genannten Variablen angepasst werden.
Die Kurzbeschreibungen und die Beschreibungen aller Variablen, die im Standard Template vorhanden sind, ist in \cite{IBM.2019} zu finden.
\begin{table}
\centering
\begin{tabularx}{\textwidth}{X|X}
Variablenname & Kurzbeschreibung \\
\hline
DFH\_REGION\_SEC & Legt fest, ob für das CICS Sicherheit im Allgemeinen aktiviert ist. \\
\hline
DFH\_REGION\_SECPRFX & Wenn DFH\_REGION\_SEC gesetzt ist, legt den Namen Perfix bei Authentificationanfragen für Ressourcen fest. \\
\hline
DFH\_REGION\_APPLID & Applikations ID der zu provisionierenden CICS Instance. \\
\hline
DFH\_LE\_HLQ & High-level qualifier\footnote{Erste Zeichen eines Dateinames, wird zum Filtern genutzt} für die Sprachumgebung\footnote{Grundeinstellungen der Programmiersprachen COBOL, PL1 und C. Mitgelieferte IBM Grundmodule} \\
\hline
DFH\_REGION\_HLQ & High-level qualifier für die CICS Dateien.\\
\hline
DFH\_REGION\_LOGSTREAM & Legt fest, wie die Log Dateien für das provisionierte CICS erstellt werden sollen. \\
\hline
DFH\_STC\_ID & User ID mit dem die CICS Instanz startet. \\
\hline
DFH\_REGION\_DFLTUSER & Default User ID für das CICS. \\
\hline
DFH\_REGION\_VTAMNODE & Name des VTAM Knotens, wenn das CICS hochfährt. \\
\hline
DFH\_REGION\_MEMLIMIT & Dem CICS maximal zur Verfügung stehender Speicherplatz. \\
\hline
DFH\_ZOS\_PROCLIB & Datei auf dem Großrechner, die den Job enthält, der für das Erzeugen der CICS Instanz zuständig ist. \\
\hline
DFH\_ZOS\_VSAM\_VOLUME & Speichersystem auf welchem die Dateien gespeichert werden sollen. Entscheidung kann auch an das System abgeben werden. \\
\hline
DFH\_CICS\_USSHOME & Homeverzeichnes des Unix System Services \\
\hline
DFH\_CICS\_HLQ & High-level qualifier von dem CICS Installationsort. \\
\end{tabularx}
\caption{Zu verändernde Variablen im minimalen CICS Template}
\label{tab:cgsvars}
\end{table}
Das diese Änderungen greifen, muss das Template aktualisiert werden.
Dies ist in der Oberfläche per Knopfdruck durchgeführt worden.

Als nächster Schritt wurde ein Testlauf und somit ein erster Versuch das Template zu provisionieren durchgeführt.
Hierbei trat beim ersten Step, der einen Job starten wollte, ein Fehler auf.
Nämlich um einen Rechte Verstoß bezüglich des Jobnames.
Bei der DATEV eG benötigt eine User ID die Rechte, um Jobs mit bestimmten Namen starten zu dürfen.
Da im Template von der IBM vorgeschlagene Jobnamen verwendet werden, kommt es zum Verstoß.
Um dieses Problem zu lösen, wurden die Jobnamen innerhalb des gesamten Templates an DATEV eG Standards angepasst.
Nachdem das Template aktualisiert wurde, wurde erneut versucht zu provisionieren.
Dabei stellte sich heraus, dass der Befehl, um ein CICS zu starten innerhalb der DATEV eG einen weiteren Parameter benötigt.
Dieser wurde hinzugefügt und danach funktionierte das Provisionieren und alle definierten Aktionen des minimalen IBM Standard CICS Templates.

\subsubsection{DATEV eG spezifischen CICS Template}
Nachdem das minimale IBM Standard CICS Template funktionsfähig war und erste Erfahrungen mit z/OSMF gesammelt wurden, wurde ein allgemeines mitgeliefertes Template untersucht.
Wie in der Tabelle \ref{vglTemps} dargestellt ist, ist dieses Template mit insgesamt 76 verwendeten Dateien sehr umfangreich.
Zu diesen Dateien zählen alle, die direkt mit dem Template in Verbindung stehen.
Das Template beinhaltet nicht nur die Möglichkeit verschiedene CICS Typen zu provisionieren, sondern auch, ob dies mit Skripten oder mit der REST-Api geschieht.
Dadurch verliert das Template an Übersichtlichkeit.
Zusätzlich kommt am Ende bei der Provisionierung keine CICS Instanz heraus, die einer DATEV eG spezifischen Instanz entspricht.
Auf Grund dessen wurden alle für ein DATEV eG spezifischen CICS nicht notwendigen Dateien entfernt.
Wie in der Tabelle \ref{vglTemps} zu sehen ist, hatte das unter anderem die Löschung von knapp der Hälfte der Dateien zur Folge.
Des Weiteren wurden auch viele nicht benötigte Variablen und Steps entfernt.
Dadurch schrumpft die provision.xml um circa ein Drittel.
Zusammenfassend lässt sich sagen, dass das Template an Übersichtlichkeit gewonnen hat.

Als nächstes wurde mit der Modifizierung der restlichen Dateien begonnen.
Als Ziel davon steht eine funktionsfähige DATEV eG spezifische CICS Instanz.
Zunächst wurden die Namen der CICS Dateien\footnote{Beschreibung in Absatz \ref{sssec:speziDat}} an die DATEV eG internen Namenskonventionen angepasst.

In Zusammenarbeit mit den CICS Administratorenteam wurde festgelegt, dass eine jede provisionierte CICS Instanz ihre eigene CSD Datei zur Verfügung gestellt bekommen soll.
Hierfür soll die bestehende von den Kollegen gepflegte Datei kopiert und mit bestimmten Namenkonventionen gespeichert werden.
Somit ist sichergestellt, dass durch die neu provisionierten Instanzen die Alten nicht beeinflusst werden.
Außerdem kann jeder Anwender so ohne Seiteneffekte seine CSD bearbeiten.
Ein weiterer Vorteil ist, dass bei der Deprovisionierung diese Kopie der Standard Datei ohne Nebenwirkungen gelöscht werden kann.
Dadurch dass die Datei, die von den Kollegen gepflegt wird als Grundlage verwendet wird, sind alle provisionierten Instanzen immer auf dem aktuellsten Stand.
Um dies Umzusetzen musste ein JCL Job geschrieben werden, der den Kopiervorgang abbildet.
Anschließend wurde dieser mittels eines neuen Steps in den Workflow eingebunden.
Außerdem mussten bestimmte Gruppen zu der CSD Liste der CICS Instanz hinzugefügt werden.
Die JCL ist in Abbildung \ref{code:addCSD} abgebildet.
Die Reihenfolge ist relevant, da es der Initialisierungsreihenfolge entspricht.

die JCL genau erkären..... bzw job im grundlagen teil erkären

Ein spezielles Augenmerk lag auf der Editierung der `createCICS.jcl`-Datei.
In dieser befindet sich die Definition des STC Jobs für das provisionierte CICS.
Im Standard IBM Template beinhaltet diese zunächst ein Makro für die Validierung von den SIT Parametern.
Noch bevor die Jobdefinition beginnt, werden alle aus der Datei für die Eingabevariablen benötigten Variablenwerte in temporäre Zwischenvariablen eingefügt.
Dadurch ist eine Änderung nur an einer Stelle notwendig, falls sich etwas an der Variablen ändert.
Danach folgt die Definition des Jobs, diese setzt sich aus folgenden Hauptbestandteilen zusammen:\\
\begin{itemize}
\item Einbindung der benötigten Bibliotheken
\item Einbindung der zuvor angelegten CICS spezifischen Dateien
\item Definition der SIT Parameter
\end{itemize}
In Abbildung \ref{HIER ABBILDUNG mit einlese von sitparams} ist zu sehen, dass es vor allem bei der Definition der SIT Parameter zu tief verschachtelten if-Bedingungen kommen kann.
Es handelt sich um den Code, der für das Einlesen der Variable `DFH\_REGION\_SITPARAMS` aus der Eingabedatei zuständig ist.
In dieser Variablen werden die SIT Parameter als Komma separierter String angegeben.
Für die Erzeugung eines DATEV eG spezifischen CICS, wurde das Makro für die Validierung von SIT Parametern beibehalten.
Alles danach wurde zunächst durch eine zur Verfügung gestellten DATEV eG Standard JCL, für die Erzeugung eines CICS, ersetzt.
Nach und nach ist die Logik, wie die aus Abbildung \ref{gleiche abbildung wie oben}, hinzugefügt worden.
Zusätzlich wurden die vorher statische DATEV eG Standard JCL durch Verwendung der Templatevariablen dynamisiert.

Zu der Definition der SIT Parameter ist zu sagen, dass hier nur die wirklich benötigten mit aufgenommen wurden.
Die anzunehmenden Werte wurden einzeln mit den CICS Administratorenteam besprochen und festgelegt.
Es ist zu beachten, dass es im IBM Standard Template zwei Möglichkeiten gibt, die Parameter zu setzen.
Für bestimmte SIT Parameter besteht eine Variable innerhalb des Templates.
Für alle anderen ist die Variable `DFH\_REGION\_SITPARAMS` vorgesehen.
In dieser Arbeit wurde hauptsächlich auf letztere Möglichkeit gesetzt.
Dadurch sind die SIT Parameter nur an einer Stelle im Template zu verwalten beziehungsweise die Verwaltung wird nicht auf zwei Arbeitsweisen verteilt.

Schließlich hat die Provisionierung eines DATEV eG spezifischen CICS Instanz funktioniert.
Dies wurde mit einem Anmeldevorgang an diese CICS sichergestellt.
Außerdem sind alle Standard DATEV eG Transaktionen funktionsfähig.
Es wurden alle Dateien wieder pflichtgerecht gelöscht.

\subsubsection{Bereitstellung Db2}\label{sssec:db2tpl}
In diesem Absatz wird die Provisionierung deiner Db2 Datenbank beschrieben.
Da die Systemumgebung noch der Testplex ist, nur die Datenbank ohne Tabelle, ohne Daten.

Für die Erstellung einer Db2 Datenbank existiert innerhalb der DATEV eG eine REST-API.
Wie im Absatz \ref{sssec:workflow} beschrieben, ist es möglich innerhalb eines Workflow Steps einen REST-Request abzusenden.
Der Code ist in Abbildung \ref{code:db2prov} zu sehen.
So muss im Body des Requests unter anderem der Datenbankname und eine UserID übergeben werden.
Der Code für das Löschen der Datenbank sieht ähnlich aus, nur handelt es sich um einen DELETE-Request.
So wurden zwei neue Steps erzeugt und in den Workflow eingebunden.

Die API ist nur dazu fähig Datenbanken auf einem bestimmten Datenbanksystem zu erzeugen.
Um die Datenbank aus der CICS Instanz heraus nutzen zu können, muss dem CICS dieses Datenbanksystem mitgeteilt werden.
Hierfür ist, wie in Abbildung \ref{code:addCSD} bereits zu sehen ist, das Hinzufügen einer weiteren CSD Gruppe notwendig.
Des Weiteren müssen weitere Bibliotheken in der `createCICS.jcl` aufgenommen werden.
Um den Aufruf möglichst dynamisch zu gestalten wurden, zusätzlich neue Variablen im Template definiert.
Diese werden in der Eingabedatei des Templates gesetzt.

In Abilldung zeile SOUNSSO

\subsubsection{Bereitstellung MQ}
In diesem Absatz wird die Provisionierung einer MQ Queue beschrieben.
Es ist auch möglich einen MQ Queuemanager zu provisionieren, der Fokus dieser Arbeit liegt aber auf der Bereitstellung von Queues.
Bei einem Queuemanager handelt es sich um ein Subsystem, deshalb wird vorerst von einer automatischen Bereitstellung abgesehen.
Auf dem Testplex wird des Weiteren die benötigte Funktion, dass eine CICS Transaktionen über eine Queue gestartet wird, nicht untersucht.

Da, wie im Absatz \ref{rechArch} beschrieben, sehr viele gleichartige Queues benötigt werden, wurde für die Erstellung dieser von den MQ Administratorenteam ein Rexx Skript angefertigt.
Dieses Skript steht dieser Arbeit zur Verfügung.
Es wurde mit Hilfe von neu erstellten Templatevariablen dynamisiert.
Anschließend wurde es in den Provisionierungsworkflow mit Hilfe eines neuen Steps aufgenommen.
Für die Deprovisionierung besteht noch kein Script.
Hierfür wurde auf Grundlage des Provisionierungsskriptes ein Eigenständiges erzeugt und ebenfalls in das Template aufgenommen.

Ähnlich wie in Absatz \ref{sssec:db2tpl} für die Datenbank beschrieben, muss der CSD Datei eine weitere Gruppe für den Queuemanager angegeben werden.
Dadurch hat die CICS Instanz auf alle Queues, die sich innerhalb dieses Managers befinden, Zugriff.
Des Weiteren ist die Aufnahme weiterer Bibliotheken in der `createCICS.jcl` notwendig.

\section{Entwicklungssystemumgebung}
Innerhalb der Entwicklungssystemumgebung sind die Sicherheits- und Rechtevorschriften schärfer als auf dem Testplex.
So wäre es zwar möglich alle für die administrativen Aufgaben notwendigen Rechte einer persönlichen UserID zu geben.
Dies würde bedeuten, dass alle Anwender dieses Templates diese Rechte auch benötigen.
Dies würde zu einem Chaos auf dem System führen, da sie auch außerhalb des Templates diese Rechte besitzen würden.
Somit wurde in Absprache mit den Administratorenteams für CICS und MQ festgelegt hierfür jeweils einen technischen User\footnote{User ID mit zunächst keinen Berechtigungen} zu beantragen.
Diesem werden nur die benötigten Rechte übergeben und ist somit sehr anwendungsspezifisch.
Um als Anwender das Template nutzen zu können, werden nur die Rechte benötigt Jobs mit diesen technischen Usern ausführen zu dürfen.
Für Db2 ist ein solcher User nicht notwendig, da das Datenbanksystem hinter der REST-API für alle zugänglich ist und jeder darauf Datenbanken erstellen darf.

Bei der Übertragung des Templates von der Testsystemumgebung auf die Entwicklungssystemumgebung waren in allen drei Bereichen des Templates notwendig.

\subsection{CICS Anpassung}
Zunächst wurden alle Steps modifiziert, so dass sie den CICS spezifischen technischen User verwenden.
Als nächstes musste ein Parameter bei der Erstellung der CICS spezifischen Dateien hinzugefügt werden.
Es handelt sich um den Massageclass Parameter mit dem Wert `NONE'.
Dadurch sind die Daten von täglichen Datensicherung der Entwicklungssystemumgebung ausgenommen.
Da die Dateien bei einem Deprovisioning gelöscht werden, ist keine Sicherung notwendig.
Außerdem ändert sich die CSD Datei, die als Vorlage gilt, auf die Standard Entwicklungssystemumgebung CSD Datei.
Die Db2 und MQ Bibliotheken, die das CICS anzieht, besitzen einen anderen Namen.
So musste dies in der `createCICS.jcl` angepasst werden.
Zusätzlich musste ein SIT Parameter angepasst werden, so dass die Log Dateien funktionisfähig sind.

\subsection{Db2 Anpassung}
Für die Datenbank Provisionierung waren keine Änderungen am Template notwendig.
TABELLEN UND DATEN ABER SCHON

\subsection{MQ Anpassung}
Bei der grundsätzlichen Provisionierung der Queues gab es keine Veränderung.
Jedoch wird ein anderer Queuemanager benötigt, dies zieht eine Änderung der Gruppe in der CSD nach sich.
Da eine der Queues, jedoch die Transaktion und somit die Anwendung starten soll, sind sogenannte MQ Prozesse notwendig.
Diese wurden in das Script aufgenommen.

Das diese funktionieren, muss CICS seitig eine Queue vorhanden sein, die sich um diese speziellen Fälle kümmert.
Jede CICS Instanz benötigt somit eine spezifische Queue, dies muss in der MQ CSD Gruppe hinterlegt werden.
In diesem Schritt wurde beschlossen, die Verwaltung der MQ CSD Gruppe komplett dem Template zu übergeben.
Diese Entscheidung hatte eine Änderung des in Abbildung \ref{code:addCSD} gezeigten Codes zu Folge.
So wird, wie in Abbildung \ref{code:createGrp} abgebildet, zunächst eine Gruppe angelegt und erst anschließend dem CSD hinzugefügt.

\section{Nutzwertanalyse}
