\thispagestyle{empty}
\section*{Kurzdarstellung}
\label{sec:kurzdarstellung}
Ziel dieser Arbeit ist es, zu bestimmen, ob die Bereitstellung von Laufzeitumgebungen für legacy z/OS Anwendungen über einen cloud nativen, Platform-as-a-Service Ansatz bei DATEV e.G. möglich ist.
Es werden folgende Forschungsfragen gestellt:
\begin{itemize}
\item Ist es möglich, den Bereitstellungsprozess für z/OS Anwendung bei DATEV e.G. mit Hilfe des \glqq IBM Cloud Provisioning and Management for z/OS\grqq-Tools an cloud native Prozesse ("Self Service") anzunähern?
\item Erzeugt die Nutzung von \glqq IBM Cloud Provisioning and Management for z/OS\grqq{} einen Mehrwert bei den Stakeholdern, also den Entwicklerteams und den Administratorenteams?
\end{itemize}

Dafür wurde anhand einer Beispielanwendung von DATEV das Tool \glqq IBM Cloud Provisioning and Management for z/OS\grqq untersucht.
Es wurden zwei vorhandene Möglichkeiten aufgezeigt, eine davon implementiert.
Für ein Meinungsbild bezüglich  Mehrwertes und Akzeptanz des Tools, wurden Interviews mit Stakeholdern durchgeführt.
Diese Bild zeigt, dass in dem Tool eine Chance auf Verbesserung der aktuellen Prozesse gesehen wird.

Ergebnis war auch, dass die implementierte Variante nicht optimal für den Praxiseinsatz bei DATEV e.G. ist, aber eine wichtige Basis für die automatisierte Bereitstellung von Laufzeitumgebungen für z/OS Anwendungen darstellt.
Weiterführende Forschung könnte darauf aufbauend Variante zwei untersuchen und  Möglichkeiten einer weiteren, parxisgeeigneteren  Optimierung des z/OS Bereitstellungsprozesses aufzeigen.



