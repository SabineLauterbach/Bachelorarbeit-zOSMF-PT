\thispagestyle{empty}
\section*{Kurzdarstellung}
\label{sec:kurzdarstellung}
Das Ziel dieser Arbeit ist es zu bestimmen, ob die Bereitstellung von Laufzeitumgebungen für legacy z/OS Anwendungen an den cloud nativen, Platform-as-a-Service Ansatz bei der DATEV e.G. möglich ist.
Dazu werden folgende Forschungsfragen gestellt:
\begin{itemize}
\item Ist es möglich, den Bereitstellungsprozess für z/OS Anwendung bei DATEV e.G. mit Hilfe des \glqq IBM Cloud Provisioning and Management for z/OS\grqq-Tools an cloud native Prozesse anzunähern?
\item Erzeugt die Nutzung von \glqq IBM Cloud Provisioning and Management for z/OS\grqq{} einen Mehrwert bei den Stakeholdern, also den Entwicklerteams und den Administratorenteams?
\end{itemize}

Für die Beantwortung dieser Fragen, wurde zunächst anhand einer Beispielanwendung, der DATEV- Rechnungsschreibung, \glqq IBM Cloud Provisioning and Management for z/OS\grqq untersucht.
Es wurden nur eine von zwei Möglichkeiten implementiert.
Um ein Meinungsbild bezüglich eines Mehrwertes und der Akzeptanz  des Tools zu erstellen, wurden Interviews mit den Stakeholdern durchgeführt.
Diese Bild zeigt, dass in dem Tool eine Chance auf Verbesserung des aktuellen Prozesses gesehen wird.

Es stellte sich heraus, dass die implementierte Variante nicht optimal ist, aber die automatisierte Bereitstellung von Laufzeitumgebungen für legacy z/OS Anwendungen ermöglicht.
Weiterführende Forschung könnte sich mit der Implementierung der zweiten Möglichkeit und mit den damit verbundenen weiteren Verbesserungen des Bereitstellungsprozesses beschäftigen.



