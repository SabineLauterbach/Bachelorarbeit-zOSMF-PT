\thispagestyle{empty}
\section*{Sperrvermerk}
Die vorliegende Arbeit beinhaltet interne vertrauliche Informationen der DATEV e.G.
Sie ist nur für die Beteiligten an der Begutachtung bestimmt.
Die Weitergabe des Inhalts der Arbeit im Gesamten oder in Teilen sowie das Anfertigen von Kopien oder Abschriften – auch in digitaler Form – vor dem Ablauf der Sperrfrist von 2 Jahren untersagt.
Ausnahmen bedürfen der schriftlichen Genehmigung der DATEV e.G.

\newpage
\section*{Kurzdarstellung}
\label{sec:kurzdarstellung}
Ziel dieser Arbeit ist es, zu bestimmen, ob die Bereitstellung von Laufzeitumgebungen für legacy z/OS Anwendungen über einen cloud nativen Platform-as-a-Service Ansatz bei DATEV e.G. möglich ist.
Es werden folgende Forschungsfragen gestellt:
\begin{itemize}
\item Ist es möglich, den Bereitstellungsprozess für z/OS Anwendung bei DATEV e.G. mit Hilfe des \glqq IBM Cloud Provisioning and Management for z/OS\grqq-Tools an cloud native Prozesse anzunähern?
\item Erzeugt die Nutzung von \glqq IBM Cloud Provisioning and Management for z/OS\grqq{} einen Mehrwert bei den Stakeholdern, also den Entwicklerteams und den Administratorenteams?
\end{itemize}

Dafür wurde anhand einer Beispielanwendung von der DATEV e.G. das Tool \glqq IBM Cloud Provisioning and Management for z/OS\grqq{} untersucht.
Es wurden zwei vorhandene Möglichkeiten aufgezeigt, eine davon implementiert.
Für ein Meinungsbild bezüglich  Mehrwertes und Akzeptanz des Tools, wurden Interviews mit Stakeholdern durchgeführt.
Diese Bild zeigt, dass in dem Tool eine Chance auf Verbesserung der aktuellen Prozesse gesehen wird.

Ergebnis war auch, dass die implementierte Variante nicht optimal für den Praxiseinsatz bei DATEV e.G. ist, aber eine wichtige Basis für die automatisierte Bereitstellung von Laufzeitumgebungen für z/OS Anwendungen darstellt.
Weiterführende Forschungen könnte darauf aufbauend Variante zwei untersuchen und  Möglichkeiten einer weiteren, praxisgeeigneteren Optimierung des z/OS Bereitstellungsprozesses aufzeigen.

\newpage
\section*{Vorwort}
Die vorliegende Bachelorarbeit entstand im Rahmen meines Verbundstudiums bei der DATEV e.G. in Nürnberg.

Für die Betreuung meiner Bachelorarbeit möchte ich mich bei Prof. Dr.-Ing. Korbinian Riedhammer und Prof. Dr. rer. nat. Friedhelm Stappert bedanken.

Für ihre Unterstützung möchte ich auch meiner Betreuerin in der Firma DATEV e.G., Sabine Lauterbach,  herzlich danken.

Außerdem danke ich allen Kollegen aus dem CICS-,Db2-,IBM MQ-Administratorenteams und dem Entwicklerteam der DATEV-Rechnungsschreibung dafür, dass ich jeder Zeit mit Fragen und Anliegen auf sie zugehen durfte und mir immer freundlich weitergeholfen wurde.

Nicht zuletzt danke ich meiner Familie und Freunde für die Motivation und den Beistand während meines Bildungsweges, welcher mir dadurch deutlich erleichtert wurde.

