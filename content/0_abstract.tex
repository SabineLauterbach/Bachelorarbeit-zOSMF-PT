\thispagestyle{empty}
\section*{Kurzdarstellung}
\label{sec:kurzdarstellung}
Innerhalb der DATEV eG gewinnen PaaS (Plattform as a Service) Ansätze immer mehr an Bedeutung.
Ein Vorteil dabei ist die unkomplizierte, automatisierte Provisionierung von Laufzeitumgebungen.
Im Vergleich dazu ist der Bereitstellungsprozess für legacy z/OS Anwendungen mit vielen manuellen Schritten und vielen Absprachen, auch abteilungsübergreifend, verbunden.

Das Ziel dieser Forschung ist es zu bestimmen, ob die automatische Bereitstellung von Laufzeitumgebungen für legacy z/OS Anwendungen möglich ist.
Dazu werden folgende Forschungsfragen gestellt:
\begin{enumerate}
\item Ist es technisch möglich mit dem \glqq IBM Cloud Provisioning and Management for z/OS\grqq-Toolkit eine Laufzeitumgebung für legacy z/OS Anwendungen automatisiert bereitzustellen?
\item Wird dadurch der aktuelle Bereitstellungsprozess schneller und auch sicherer?
\item Erzeugt die Nutzung einen Mehrwert bei den Stakeholdern, also den Entwicklerteam und den Administratorenteams?
\end{enumerate}
Um diese Forschungsfragen zu beantworten, wurde zunächst anhand einer Beispielanwendung, der DATEV Rechnungsschreibung, das \glqq IBM Cloud Provisioning and Management for z/OS\grqq-Toolkit untersucht.
Das Toolkit bietet zwei Möglichkeiten für die automatisierte Bereitstellung von Laufzeitumgebungen.
Es wurde sich für eine dieser Möglichkeiten entschieden, diese wurde implementiert.\\
Anschließend wurde der dadurch ermöglichte Prozess aufgezeigt und mit dem etablierter Prozess verglichen.
Dabei stellte sich heraus, dass der ermöglichte Bereitstellungsprozess schneller und durch weniger Absprachen auch weniger fehleranfällig ist.
Dennoch ist die Lösung nicht optimal und das Toolkit bietet weitere Möglichkeiten zur Verbesserung.\\
Schließlich sind Interviews mit den Stakeholdern bezüglich eines Mehrwertes des neuen Prozesses durchgeführt worden.
Sowohl die befragten Entwickler als auch die befragten Administratoren sehen in dem Toolkit eine Chance auf Verbesserung des aktuell etablierter Bereitstellungsprozesses.
Jedoch ist die momentane Lösung zwar funktionsfähig, aber noch bezüglich firmenweiten und einfacheren Einsatz zu optimieren.

Auf dieser Grundlage lässt sich sagen, dass das \glqq IBM Cloud Provisioning and Management for z/OS\grqq-Toolkit die automatisierte Bereitstellung von Laufzeitumgebungen für legacy z/OS Anwendungen ermöglicht.
Die in dieser Arbeit implementierte Möglichkeit ist nicht optimal, bietet aber bereits einen Mehrwert für die Stakeholder.
Weiterführende Forschung könnte sich mit der Implementierung der zweiten Möglichkeit und mit den damit verbundenen weiteren Verbesserungen des Bereitstellungsprozesses beschäftigen.



