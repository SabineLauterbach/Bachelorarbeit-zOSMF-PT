\thispagestyle{empty}
\section*{Kurzdarstellung}
\label{sec:kurzdarstellung}
Bei DATEV eG gewinnen Cloud Native und PaaS (Plattform as a Service) Ansätze immer mehr an Bedeutung.
Ein Vorteil dabei ist die automatisierte Provisionierung von Laufzeitumgebungen (Self Services).
Für legacy z/OS Anwendungen ist das Bereitstellen von Laufzeitumgebungen ein manueller, fehleranfälliger Prozess mit vielen Abstimmungen.

Ziel dieser Forschung ist es, zu bestimmen, ob die automatisierte Provisionierung auch für legacy z/OS Anwendungen möglich ist.
Folgende Forschungsfragen werden gestellt:
\begin{enumerate}
\item Ist es technisch möglich, mit dem \glqq IBM Cloud Provisioning and Management for z/OS\grqq-Toolkit Laufzeitumgebungen für legacy z/OS Anwendungen automatisiert bereitzustellen?
\item Wird der Bereitstellungsprozess dadurch schneller und sicherer?
\item Erzeugt er einen Mehrwert bei Stakeholdern (Entwickler, Administratoren)?
\end{enumerate}
Um diese Forschungsfragen zu beantworten, wurde sich für eine von zwei technischen Alternativen im Toolkit entschieden. 
Mit dieser wurde beispielhaft für eine DATEV-Anwendung ein automatisierter Prozess implementiert.
Die gewählte Alternative ist jedoch nicht optimal, die zweite Alternative im Toolkit bietet Möglichkeiten zur Verbesserung.\\
Der Prozess wurde aufgezeigt, mit dem etablierten Vorgehen verglichen, und in Interviews mit den Stakeholdern bewertet.

Sowohl befragte Entwickler als auch Administratoren sehen in der Lösung grundsätzlich eine Verbesserung des aktuell etablierter Bereitstellungsprozesses.
Sie ist aber  insb. für den firmenweitem Einsatz noch zu optimieren.

Damit wird das \glqq IBM Cloud Provisioning and Management for z/OS\grqq-Toolkit für die automatisierte Bereitstellung von Laufzeitumgebungen für legacy z/OS Anwendungen als geeignet erachtet.
Die in dieser Arbeit implementierte Variante ist nicht optimal, bietet aber bereits einen Mehrwert für die Stakeholder.
Weiterführende Forschung könnte sich mit der Implementierung der Alternative und den damit verbundenen weiteren Verbesserungen des Bereitstellungsprozesses beschäftigen. 



